Som en av de mest brukte statistiske metodene, har regresjonsmodeller en fundamental posisjon i statistikk. En nøkkeldel av regresjonsanalysen er å skaffe inferens om kovariatene som brukes til å modellere responsvariabelen, og ofte tilegne kovariatene en \textit{relativ viktighet}, for å kvantifisere, eller rangere, deres bidrag til den statistiske modellen. For å gjøre dette, eksisterer flere metoder fra ulike perspektiver. Til tross for mange forskjellige metoder, har det ikke blitt oppnådd en konsensus, og den tradisjonelle fremgangsmåten med $p$-verdier har skapt en reproduserbarhetskrise i samfunns- og biomedisinsk forskning. Vårt bidrag for å bøte på dette, er å foreslå en Bayesiansk metode for å beregne relativ variabelviktighet. Denne metoden er designet for at forskere skal tolke den statistiske modellen og dens resultater grundigere, i stedet for å slå seg til ro med konklusjoner basert på en forhåndsbestemt terskel.
\\
\\
Vår metode, betegnet som \textit{Bayesiansk Variabel Viktighet} (BVV), er implementert ved å overføre logikken fra mer etablerte, frekventistiske metoder, til det Bayesianske rammeverket. BVV er anvendbart på generaliserte lineære blandingsmodeller (GLBM) som har kontinuerlige, binomiske og Poisson fordelte responser. Kjernen i metoden er å benytte relativ vekting på kovariatene før en Bayesiansk GLBM konstrueres. Dette produserer posteriore fordelinger av den relative viktigheten til alle kovariatene i modellen, samt de estimerte fordelingene til den marginale og betingede $R^2$. For å gjøre metodikken lett tilgjengelig for forskere på tvers av fagfelt, ble en R pakke kalt \texttt{BayesianVariableImportance} lagd.
\\
\\
Basert på forfatterens tidligere verk for lineære blandingsmodeller \citep{Arnstad:Relative_variable_importance_in_Bayesian_linear_mixed_models:2024}, simulasjonsstudier, case studier og en anvendelse på reelle data, har vi vist at BVV metoden er en levedyktig analog til eksisterende frekventistiske metoder. Metoden er i stand til å produsere plausible resultater for GLBM med komplekse kovariansstrukturer, samtidig som den er beregningsmessig effektiv. Forhåpentligvis kan BVV metoden bli brukt på tvers av ulike fagfelt og hjelpe forskere i deres arbeid. Med tanke på at relativ variabelviktighet er et område av stor interesse og aktiv forskning, nylig også i det Bayesianske rammeverket, tror vi at BVV metoden kan bli ytterligere forbedret i fremtiden.
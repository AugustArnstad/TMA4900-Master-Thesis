The goal of this thesis was to provide a novel variable importance measure in the Bayesian framework for generalized linear mixed models. To do so, we applied the relative weights method and fit a Bayesian GLMM. Then, we extended a simple definition of the $R^2$ for GLMMs into the Bayesian framework to obtain our proposed definition. The posterior distribution of the Bayesian GLMM is sampled, before the $R^2$ is decomposed and distributed to the covariates, to allocate them a relative importance. The methodology is named the Bayesian Variable Importance (BVI) method and wrapped in an R package. 
\\
\\
From simulation studies, case studies and real world applications, it has been shown that the BVI method is capable of providing plausible and robust estimates. The uncertainty in estimates is quantified, and the method allows researchers to carry out comprehensive inference. Being a general method, the BVI method can be applied to a wide range of regression models, and has proven to be computationally efficient. It is available to any reader with access to the statistical software R, and has many areas of applications across sciences. There is much potential for further augmentation of the method, both theoretically and practically. It is our aspiration, that the BVI method provides a useful tool, and that it can provide researchers with more inference on their statistical models.
% drive further research in the field of variable importance measures.
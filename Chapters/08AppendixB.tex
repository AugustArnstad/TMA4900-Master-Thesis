
\begin{lstlisting}[language=R, caption=Usage of the BayesianImpGLMM package with plots and examples.]
## GENERAL SETUP
First, we set up the necessary libraries and configure the environment for our analysis. This includes loading essential packages and setting options for chunk output and plot dimensions."



## GENERAL SETUP
First, we set up the necessary libraries and configure the environment for our analysis. This includes loading essential packages and setting options for chunk output and plot dimensions."

```{r setup, input=FALSE, echo=FALSE}
library(formatR)
showsol <- FALSE
library(knitr)
library(devtools)
knitr::opts_chunk$set(tidy.opts = list(width.cutoff = 68), 
                      tidy = TRUE, 
                      warning = FALSE, 
                      error = FALSE, 
                      message = FALSE, 
                      echo = TRUE, 
                      fig.width=7, 
                      fig.height=5, 
                      fig.align="center")
```

## INSTALLING THE PACKAGE
This section ensures the devtools package is installed, which is required for installing packages from GitHub. We then install the BayesianVariableImportance package directly from GitHub using devtools::install_github(). In the package under the Hello.R file, all functions are defined with corresponding documentation.
```{r}
# If not already installed, install the 'devtools' package
if(!require(devtools)) install.packages("devtools")
devtools::install_github("AugustArnstad/BayesianVariableImportance")
library(BayesianVariableImportance)
```

## SIMULATE DATA
In this part, we simulate data to demonstrate the functionality of the BayesianVariableImportance package. We generate random variables used as fixed effects with different correlation structures and random effects. Note that the coefficients used here are a bit large for the Poisson model, consider lowering them. The data is then structured into data frames for further analysis. If you have a suitable dataset you can use this instead.

```{r}
library(remotes)
library(INLA)
library(mnormt)
library(ggplot2)
library(reshape2)
library(RColorBrewer)
library(tidyr)
library(dplyr)

set.seed(1)

simulate_data <- function(n = 10000, n_groups = 100, covariance_level=0) {
  # Simulate fixed effects
  
  sigma <- matrix(c(1, covariance_level, covariance_level, 
                    covariance_level, 1, covariance_level, 
                    covariance_level, covariance_level, 1), 3, 3)
  
  X <- MASS::mvrnorm(n = n, mu = c(0, 0, 0), Sigma = sigma)
  X1 <- X[, 1]
  X2 <- X[, 2]
  X3 <- X[, 3]

  # Simulate random effects groups
  Z1 <- sample(1:n_groups, n, replace = TRUE)
  random_effect_contributions_z1 <- rnorm(n_groups, mean = 0, sd = 1)[Z1]
  # Z2 <- sample(1:(n_groups/10), n, replace = TRUE)
  # random_effect_contributions_z2 <- rnorm(n_groups/10, mean = 0, sd = 1)[Z2]

  # Coefficients for fixed effects
  beta1 <- 1
  beta2 <- sqrt(2)
  beta3 <- sqrt(3)

  # Linear predictor
  eta <- beta1*X1 + beta2*X2 + beta3*X3 + random_effect_contributions_z1 #+ random_effect_contributions_z2 

  # Binomial with logit link
  p_logit <- exp(eta) / (1 + exp(eta))
  y_logit_bin <- rbinom(n, size = 1, prob = p_logit)
  data_logit <- data.frame(y_logit_bin, X1, X2, X3, Z1)

  # Binomial with probit link
  p_probit <- pnorm(eta)
  y_probit_bin <- rbinom(n, size = 1, prob = p_probit)
  data_probit <- data.frame(y_probit_bin, X1, X2, X3, Z1)

  # Poisson with log link
  lambda <- exp(eta)
  y_pois <- rpois(n, lambda = lambda)
  data_poisson <- data.frame(y_pois, X1, X2, X3, Z1)
  
  epsilon = rnorm(n, mean=0, sd=sqrt(1))
  y_normal <-  beta1*X[, 1] + beta2*X[, 2] + beta3*X[, 3] + random_effect_contributions_z1 + epsilon 
  data_normal <- data.frame(y_normal, X1, X2, X3, Z1)
  
  
  list(binomial_logit = data_logit, 
       binomial_probit = data_probit, 
       poisson = data_poisson, 
       normal = data_normal)
}



```


## USAGE
Here we demonstrate the usage of the BayesianVariableImportance package. We fit Bayesian binomial, Poisson and gaussian models and sample posterior distributions for different simulated datasets using functions from the package. Then, plots are made to display the results.
```{r}
set.seed(1234)

datasets <- simulate_data()

glmm_logit <- y_logit_bin ~ X1 + X2 + X3 + f(Z1, model="iid", hyper=list(prec = list(
        prior = "pc.prec",
        param = c(1, 0.01),
        initial = log(1)
      ))
)

glmm_pois <- y_pois ~ X1 + X2 + X3 + f(Z1, model="iid", hyper=list(prec = list(
        prior = "pc.prec",
        param = c(1, 0.01),
        initial = log(1)
      ))
)

lmm <- y_normal ~ X1 + X2 + X3 + f(Z1, model="iid", hyper=list(prec = list(
        prior = "pc.prec",
        param = c(1, 0.01),
        initial = log(1)
      ))
)
    
model_logit <- BayesianVariableImportance::perform_inla_analysis(datasets$binomial_logit, glmm_logit, family = "binomial", link_func = "logit")
model_pois <- BayesianVariableImportance::perform_inla_analysis(datasets$poisson, glmm_pois, family = "poisson", link_func = "log")
model_normal <- BayesianVariableImportance::perform_inla_analysis(datasets$normal, lmm, family = "gaussian", link_func = "identity")


imp_logit <- BayesianVariableImportance::extract_importances(model = model_logit, 
                                                  data = datasets$binomial_logit,
                                                  random_names=c("Z1"), 
                                                  fixed_names=c("X1", "X2", "X3"))


imp_pois <- BayesianVariableImportance::extract_importances(model_pois, 
                                datasets$poisson,
                                random_names=c("Z1"), 
                                fixed_names=c("X1", "X2", "X3"))

#One can also use the dist_factor argument to specify the distribution factor one wishes to use
imp_pois_2 <- BayesianVariableImportance::extract_importances(model_pois, 
                                datasets$poisson,
                                random_names=c("Z1"), 
                                fixed_names=c("X1", "X2", "X3"),
                                dist_factor = log(1 + 1/exp(summary(model_pois)$fixed[1] + 0.5)))


imp_lmm <- BayesianVariableImportance::extract_importances(model_normal, datasets$normal,
                                random_names=c("Z1"), 
                                fixed_names=c("X1", "X2", "X3"))


samples_logit <- BayesianVariableImportance::sample_posterior_count(model_logit, glmm_logit, datasets$binomial_logit, n_samp=5000, additive_param = "Z1")
samples_pois <- BayesianVariableImportance::sample_posterior_count(model_pois, glmm_pois, datasets$poisson, n_samp=5000, additive_param = "Z1")
samples_lmm <- BayesianVariableImportance::sample_posterior_gaussian(model_normal, lmm, datasets$normal, n_samp=5000, additive_param = "Z1")

plots_logit <- BayesianVariableImportance::plot_samples(samples_logit)
plots_pois <- BayesianVariableImportance::plot_samples(samples_pois)
plots_lmm <- BayesianVariableImportance::plot_samples(samples_lmm)



```

## IMPORTANCES
The simplest way of obtaining the importances can be done by looking at these objects. Note that these are sampled, so they do not represent the mean of the samples used for plotting further down.
```{r}
imp_logit

imp_pois

imp_lmm
```

## PLOTS
These are the default plots that are implemented in the package, displaying the importance of all effects and $R^2$ metrics.
```{r}
plots_logit$fixed_effects
plots_logit$random_effects
plots_logit$heritability
plots_logit$R2

plots_pois$fixed_effects
plots_pois$random_effects
plots_pois$heritability
plots_pois$R2

plots_lmm$fixed_effects
plots_lmm$random_effects
plots_lmm$heritability
plots_lmm$R2

```


## CUSTOM PLOT
Cutsomizing plots is often very nice to display information in the way you want it. Therefore, we show how one can customize the plots using ggplot2 based on the samples drawn.
```{r}
random <- "Z1"

random_plot <- ggplot(samples_pois$scaled_random_samples, aes(x = !!sym(random))) +
  geom_histogram(aes(y = ..density..), fill = "#C6CDF7", alpha = 0.7, bins = 40, color = "black") +
  geom_density(color = "#E6C6DF", adjust = 1.5, linewidth=1.5) +
  geom_point(aes(x = mean(samples_pois$scaled_random_samples$Z1), y = 0), color = "#E6C6DF", size = 4) +
  labs(#title = paste("Heritability of mass"),
       x = "Samples of relative importance of random effect",
       y = "Frequency") +
  theme_minimal() +
  theme(legend.position = "none",
    axis.title.x = element_text(size = 24),
    axis.title.y = element_text(size = 24),
    axis.text.x = element_text(size = 24),
    axis.text.y = element_text(size = 24)
    ) 

random_plot

str(samples_pois)

# Assuming 'samples_pois$scaled_importance_samples' is your dataframe
data_long <- samples_pois$scaled_importance_samples %>%
  pivot_longer(cols = c(X1, X2, X3), names_to = "Variable", values_to = "Value")

# Updated plot code
fixed_plot <- ggplot(data_long, aes(x = Value)) +
  geom_histogram(aes(y = ..density..), fill = "#C6CDF7", alpha = 0.7, bins = 40, color = "black") +
  geom_density(color = "#E6C6DF", adjust = 1.5, linewidth=1.5) +
  facet_wrap(~ Variable, scales = "free_x") +
  labs(x = "Samples of relative importance of random effect",
       y = "Frequency") +
  theme_minimal() +
  theme(legend.position = "none",
        axis.title.x = element_text(size = 24),
        axis.title.y = element_text(size = 24),
        axis.text.x = element_text(size = 24),
        axis.text.y = element_text(size = 24))

# Print the plot
fixed_plot

r2_data <- data.frame(
  Marginal_R2 = samples_pois$R2_marginal$`Marginal R2`,
  Conditional_R2 = samples_pois$R2_conditional$`Conditional R2`
)

# Reshape the data from wide to long format
r2_long <- pivot_longer(r2_data, cols = c(Marginal_R2, Conditional_R2),
                        names_to = "R2_Type", values_to = "Value")

# Create the plot
r2_plot <- ggplot(r2_long, aes(x = Value, fill = R2_Type)) +
  geom_histogram(aes(y = ..density..), alpha = 0.7, bins = 40, color = "black") +
  geom_density(adjust = 1.5, color = "black", alpha = 0.7) +
  labs(x = "R2 Values", y = "Density") +
  scale_fill_manual(values = c("Marginal_R2" = "#C6CDF7", "Conditional_R2" = "#E6C6DF")) +
  theme_minimal() +
  theme(legend.title = element_blank(),
        legend.position = "top",
        axis.title.x = element_text(size = 14),
        axis.title.y = element_text(size = 14),
        axis.text.x = element_text(size = 12),
        axis.text.y = element_text(size = 12))

# Print the plot
r2_plot

```
\end{lstlisting}
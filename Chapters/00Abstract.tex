As one of the most widely used statistical methods, regression models have a fundamental position in statistics. Obtaining inference on the covariates used to model the response is a key part of regression analysis, and often it is desirable to assign the covariates with a \textit{relative importance}, in order to quantify, or rank, their impact on the statistical model. To do so, numerous methods from multiple perspectives exist. Despite this, a consensus has not been reached, and the traditional methods using $p$-values have created a reproducibility crisis in the social and biomedical sciences. Our contribution to help remedy this, is to suggest a Bayesian relative variable importance method. The method is designed to make researchers more thoroughly interpret the statistical model and its results, rather than blindly following a threshold to draw conclusions.
\\
\\
Our method, denoted as \textit{Bayesian Variable Importance} (BVI), is implemented by transferring the logic of more established, frequentist methods, to the Bayesian framework. The BVI method is applicable to generalized linear mixed models (GLMMs) with continuous, binomial and Poisson distributed responses. The core of the method is to utilize the relative weights method on the covariates of before fitting a Bayesian GLMM and performing calculations with respect to the Bayesian framework. This produces posterior distributions of the relative importance of all covariates present in the model, as well as the estimated distributions of the marginal and conditional $R^2$. To make the methodology easily available for researchers across fields, an R package called \texttt{BayesianVariableImportance} was made.
\\
\\
Based on the author's previous work for linear mixed models \citep{Arnstad:Relative_variable_importance_in_Bayesian_linear_mixed_models:2024}, simulation studies, case studies and a real world application, we have shown that the BVI method is a viable analogue to the existing frequentist methods. The method is able to produce plausible results for GLMMs with a complex covariance structure, while being simultaneously being computationally efficient. Hopefully, the BVI method can be used across various field and help researchers in their work. With relative variable importance being a topic of much interest and active research, recently also in the Bayesian framework, we believe that the BVI method can be further improved in the future.
Some sections in this chapter overlap with the authors project thesis \citep{Arnstad}, which lead up to the masters thesis. Following the guidelines of the Institute of Mathematical Sciences, stating that sections need not be rewritten, some sections are the same (or slightly modified) as in the project thesis. To avoid problems relating to self plagirazation and clarify what is new in this thesis, the sections that are the same as in the project thesis have been assigned a footnote with the reference to the project thesis and a brief comment.
% The sections on linear regression and linear mixed models are similar to those presented in the authors project thesis \citep{Arnstad}, which lead up to the masters thesis.
\section{Linear regression}
\label{sec:linreg}
All regression models are based on the assumption that the response variable is influenced by one or more covariates.
The relationship between the response and the covariate is assumed not to be deterministic, so we expect our modelling of the response to be influenced by some random error \citep{GLMM_book}.
This means that the response is treated as a random variable, and it is desirable to decompose the response into systematic components and random components.
\subsection{Linear regression\protect\footnote{This subsection is slightly modified from the project thesis \citep{Arnstad}.}}
Assuming that an observed response $y_i$ has a linear relationship with a covariate $x_i$ is the basis for the simple linear regression.
This can be modeled by the equation
\begin{equation}
    y_i = \beta_0 + \beta_1x_i + \varepsilon_i \ ,
\end{equation}
where $\beta_0$ is the intercept, $\beta_1$ is the slope, and $\varepsilon_i$ is the error term.
The error term, or residuals, is assumed to be normally distributed with mean zero and variance $\sigma^2$, i.e. $\varepsilon_i \sim \mathcal{N}(0, \sigma^2)$.
Generalizing to multiple covariates is straightforward by defining the $n\times p$ matrix $\mathbf{X}$ as a design matrix with the, including an intercept, $p$ covariates in the columns and the $n$ observations in the rows.
With this definition, the linear regression model can be written as
\begin{equation}
    \label{eq:linreg}
    \mathbf{y} = \mathbf{X}\boldsymbol{\beta} + \mathbf{\varepsilon} \ ,
\end{equation}
where now $\mathbf{y}=(y_1, y_2, ..., y_n)$ is a vector of $n$ responses, $\boldsymbol{\beta}=(\beta_0, \beta_1, ..., \beta_{p-1})$ is a vector of coefficients including the intercept $\beta_0$, and $\mathbf{\varepsilon}=(\varepsilon_1, \varepsilon_2, ..., \varepsilon_n)$ is a vector of error terms.
The error terms are assumed to be independent and identically distributed (i.i.d.) with $\boldsymbol{\varepsilon} \sim \mathcal{N}(0, \sigma^2 \mathbf{I})$, where $\mathbf{I}$ is the identity matrix of size $n \times n$.
Consequently, the response $\mathbf{y}$ is conditionally independent given the covariates $\mathbf{X}$, i.e.
\begin{equation}
    \mathbf{y} \lvert \mathbf{X} \sim \mathcal{N}_n(\mathbf{X}\boldsymbol{\beta}, \sigma^2\mathbf{I}) \ .
\end{equation}
In practice, the coefficients $\boldsymbol{\beta}$ are estimated from the maximum likelihood estimation (MLE) method, given by
\begin{equation}
    \label{eq:beta_hat}
    \hat{\boldsymbol{\beta}} = (\mathbf{X}^T\mathbf{X})^{-1}\mathbf{X}^T\mathbf{y} \ .
\end{equation}

\subsection{Qualitative covariates}
\label{sec:qualitative}
In many cases the covariates are qualitative, meaning they are categorical variables that can be grouped into different levels or factors.
Qualitative covariates, unlike quantitative, cannot be measured numerically, and we must adjust our modelling to account for this.
A common approach to model qualitative data is to include dummy variables, which are assigned a value $1$ if the observation is in the respective category(factor) and $0$ otherwise.
Given $N$ factors, it is standard practice to model $N-1$ dummy variables and let one factor be captured by the intercept to uniquely determine the model.
Dummy encoding in this way retains the properties of the linear regression, and are limited by the same assumptions.
The model for the response $y_i$, assuming no quantitative covariates, from group $j$ with dummy encoding is then given by
\begin{equation}
    y_i =  \beta_0 + \sum_{j=1}^{N-1} \beta_j x_{i,j} + \varepsilon_i \ ,
\end{equation}
where $\beta_j$ denotes the factor coefficient of observation $i$ and the dummy variable
\begin{equation}
    x_{i,j} = \begin{cases}
        1 & \text{if observation $i$ is in group $j$} \\
        0 & \text{otherwise}
    \end{cases} \ .
\end{equation}
This way of modelling qualitative covariates is intuitive and easy to interpret, but it also assumes that factor specific effects are uniform and fixed across all levels and becomes cumbersome with many categorical covariates. 

\subsection{Correlation among covariates in linear regression}
Correlation among covariates is to be expected, as it is natural in many scenarios. However, if the correlation is very strong, this poses some serious problems when interpreting the linear regression model.
The covariates $\mathbf{x}_i$ in a linear regression are assumed to be linearly independent, so that the design matrix $\mathbf{X}$ has full rank.
If the design matrix is not of full rank, that is one or more covariates are perfectly correlated, the model \eqref{eq:linreg} is said to be \textit{multicollinear} \citep{Poole_OFarrell_1971}. 
From equation \eqref{eq:beta_hat} one can see that if the matrix $\mathbf{X}$ is not of full rank, the term $(\mathbf{X}^T\mathbf{X})^{-1}$ is not invertible and the MLE of $\boldsymbol{\beta}$ does not exist. 
Further, the variance of the MLE of $\boldsymbol{\beta}$ grows as the correlation between covariates grows \citep[p. 116]{GLMM_book}. A larger variance in $\boldsymbol{\hat{\beta}}$ also leads to larger standard errors and larger $p$-values for $\boldsymbol{\hat{\beta}}$, making it hard to assess the model.
Both coefficients and covariates affect the total marginal model variance, which can be decomposed as
\begin{equation}
    \label{eq:var_y}
    \text{Var}(\mathbf{y}) = \text{Var}(\mathbf{X}\boldsymbol{\beta}) + \text{Var}(\boldsymbol{\varepsilon}) = \boldsymbol{\beta}^T\mathbf{V}\boldsymbol{\beta} + \sigma^2_{\varepsilon} = \sum_{j=1}^p\beta_j^2v_j +\sum_{j=1}^{p-1}\sum_{k=j+1}^{p} \beta_j\beta_k\sqrt{v_jv_k}\rho_{jk} + \sigma_{\varepsilon}^2 \ ,
\end{equation}
\citep{gromping_relaimpo} where $\mathbf{V} = \text{Cov}(\mathbf{X})$ is the $p \times p$ covariance matrix of the covariates which is assumed to be positive definite, $\boldsymbol{\beta}$ is the $p \times 1$ vector of regression coefficients, $v_j$ the regressor variances for $j=1, ..., p$ found along the diagonal of $\mathbf{V}$ and $\rho_{jk}$ the inter-regressor correlations between regressor $j$ and $k$.
The middle term in \ref{eq:var_y} consist of the covariance between the covariates and the variance contribution from a single covariate is not immediately clear.

%this term makes it hard to assess the relative importance of each covariate. 
%To assign each covariate with an importance, we need to consider relative importance measures that can handle the correlation among covariates.

%\subsection{Relative variable importance \protect\footnote{This subsection is slightly modified from the project thesis \citep{Arnstad}.}} 
\section{Variable importance in linear regression models}
In a regression setting with multiple regression coefficients, it is often desirable to be able to assign each covariate with a measure of its relative importance to the model.
The relative importance of covariate $X_i$ is defined as the contribution to explained variance in the response $\mathbf{y}$ from $X_i$.
Assigning relative importance is no trivial task, as correlation among covariates poses a challenge in assessing the relative importance of each covariate.
\newline
\newline



\subsection{Relative importance measures}
\label{sec:rel_imp}
The coefficient of determination, $R^2$, is a widely used and intuitive summary statistic of goodness-of-fit and can also be used in model comparison. 
Conceptually, the $R^2$ quantifies how much variance in the response variable can be attributed to the covariates in the model. 
For the linear regression model, the $R^2$ is defined as
\begin{equation}
    \label{eq:R2}
    R^2 = 1 - \frac{(\mathbf{y}-\mathbf{X}\boldsymbol{\beta})^T(\mathbf{y}-\mathbf{X}\boldsymbol{\beta})}{(\mathbf{y}-\overline{\mathbf{y}})^T(\mathbf{y}-\overline{\mathbf{y}})} = \frac{\text{Var}(\mathbf{y}) - \sigma^2_{\varepsilon}}{\text{Var}(\mathbf{y})} \ ,
\end{equation}
where $\overline{\mathbf{y}}$ is the mean vector of responses $\mathbf{y}$. 
Instead of referring to the $R^2$ value alone, going forward this thesis will focus on decomposing of the $R^2$ value and allocate a proportion of $R^2$ to the model covariates. 
This decomposition is done in order to assess the relative importance, or variance explained, of each covariate in the model.
% Generally, the variance of a linear regression model can be stated as in equation \eqref{eq:var_y}, however in the case of 
The special case of uncorrelated covariates in $\mathbf{X}$ gives 
% we obtain  
\begin{equation}
    \label{eq:var_y_uncorrelated}
    \text{Var}(\mathbf{y}) = \sum_{j=1}^p\beta_j^2v_j + \sigma_{\varepsilon}^2 \ .
\end{equation}
% This case
% , the $R^2$ is consistent with the proportion of variance explained in the response total variance of the response variable \citep{gromping_relaimpo}. 
% This consistency 
and provides a natural decomposition of the $R^2$ in terms of contribution from each covariate, as each predictor $\mathbf{x}_i$ contributes $\beta_i^2v_i$ to the total response variance \citep{gromping_relaimpo}.
In (\ref{eq:var_y}) however, the response variance is split into three parts, the first two sums which comes from the regressors and the latter term which is the variance of the error. 
As mentioned, it is the middle term that poses the problem of assigning importance to each covariate, since it is not immediately clear how to distribute the contribution to variance from the covariance terms to each covariate.
The literature has established some conditions that relative importance measures should fulfill, so that they can be interpreted and compared in a sensible manner \citep{gromping_relaimpo}. As listed in \citet{gromping_relaimpo}, the methods should have
\begin{enumerate}
    \label{list:criteria}
    \item \textbf{Proper decomposition}: The model variance should be decomposed into shares for each regressor that sum up to the total variance, and the method shall allocate the shares to each regressor.
    \item \textbf{Non-negativity}: Each share of the variance should be non-negative.
    \item \textbf{Exclusion}: If a regressor is excluded from the model, $\beta_j=0$, its share of the variance should be zero.
    \item \textbf{Inclusion}: If a regressor is included in the model, $\beta_j \neq 0$, its share of the variance should be positive.
\end{enumerate}
% \begin{equation}
%     \label{eq:var_y_full}
%     \text{Var}(\mathbf{y}) = \sum_{j=1}^p\beta_j^2v_j + \sum_{j=1}^{p-1}\sum_{k=j+1}^{p} \beta_j\beta_k\sqrt{v_jv_k}\rho_{jk} + \sigma_{\varepsilon}^2 \ ,
% \end{equation}
% where $\beta_j$ is the $j$-th regression coefficient, $v_j$ is the variance of the $j$-th covariate, $\rho_{jk}$ is the correlation between the $j$-th and $k$-th covariate and $\sigma_{\varepsilon}^2$ is the variance of the error term.
\subsection{Naive decompositions}
\label{sec:naive_decomp}
To make it clear that some simple decompositions fail the conditions of relative importance measures, we will consider two naive approaches for decomposing the $R^2$. 
We denote the $R^2$ of a linear regression with regressors $X_1, \dots, X_p$ as $R^2(\{1, \dots, p\})$ and the relative importance of regressor $X_i$ as $\text{RI}(\{i\})$
\newline
\newline
The first naive method is to fit a model with all regressors $p$, and then fit a model with all regressors excluding regressor $i$. The relative importance of $X_i$ is then the difference $R^2(\{1, \dots, p\}) - R^2(\{1, \dots, p\} \setminus i)$.
To show how this fails the conditions of relative importance measures, an example from \citet{matre} is discussed. The example considers the simple case 
\begin{equation}
    Y=X_1+X_2 \ , \text{Var}(X_1) = \text{Var}(X_2)=1 \ , \text{Cov}(X_1, X_2)=0.9 \ .
\end{equation}
The $R^2$ of the model with both covariates is $R^2(\{1, 2\})=1$, since the covariates $X_1, X_2$ explain fully the response $Y$. Then one would expect that the importance of $X_1$ and $X_2$ is $0.5$ each, since they both explain half of the response variance.
Using the proposed decomposition, one would calculate
\begin{equation}
    \text{Ri}(\{2\}) = R^2(\{1, 2\}) - R^2(\{1\}) = 1 - \frac{\text{Cov}(Y, X_1)^2}{\text{Var}(Y)\text{Var}(X_1)} = 1- \frac{1.9^2}{3.8} \approx 0.05 \ ,
\end{equation}
where it is used that for the simple linear regression, the $R^2$ is given by the squared correlation coefficient between the response and the regressor.
By symmetry $\text{Ri}(\{1\})=\text{Ri}(\{2\})$, so the sum of the relative importances is $0.1$. However, the total explained variance of the model is $1$, so this decomposition violates the proper decomposition condition.
This decomposition only assign importances to the regressor based on the information that the regressor does not share with any other regressors. Therefore, it does not take into account the shared information and the importance estimated is too low.
\newline
\newline
Another naive decomposition would be to compare the relative importance of a model with one regressor $i$ to the empty model, \textit{i.e.} the model with no covariates. 
The empty model has an $R^2=0$ and therefore for $X_1$ in the above example we would have
\begin{equation}
    \text{Ri}(\{1\}) = R^2(\{1\}) - R^2(\{\emptyset\}) = \frac{\text{Cov}(Y, X_1)^2}{\text{Var}(Y)\text{Var}(X_1)} = \frac{1.9^2}{3.8} \approx 0.95 \ .
\end{equation}
Once more by symmetry we have $\text{Ri}(\{2\})=\text{Ri}(1)$, so the sum of the relative importances is $1.9$, violating the proper decomposition condition.
Conversely to the first naive approach, this decomposition assigns importances based on the full information contained in the regressor. Therefore it overestimates the importance of each variable, since the shared information is accounted for twice.
\newline
\newline
As we have seen from these naive approaches, the task of decomposing the $R^2$ value is far from trivial, and calls for more sophisticated methods.

\subsection{The LMG method}
\label{sec:lmg}
A method that handles correlation among covariates, and is frequently reinvented\citep{gromping_relaimpo} from different approaches, is the LMG method.
Therefore we shall discuss it, as it serves an important role as a leading method for assigning relative variable importance.  
The LMG method takes use of averaging over orders, meaning that it permutes the index set $\{1, ..., p\}$  of the regressors $(p-1)!$ times, excluding the intercept, and sequentially adds the regressors to the model for each permuted index set.
By adding regressors sequentially for each permutation, one can investigate how the importance of the regressors vary depending on what other regressors are included, which is useful when they are correlated.
This is justified by the assumption that there is no relevant ordering of the regressors in the index set \citep{kruskal_lmg_1987}.
For each regressor added, starting with none, it allocates a share of explained variance, or importance, and then adds a new regressor.
The final allocated share to the regressor is the average of the allocated shares to that regressor for all permutations of the set of regressors indices. 
This would mean that for two correlated regressors whose importance share varies depending on which is added first, would receive an averaged importance.
Averaging over orders is a statistical tradition \citep{kruskal_lmg_1987} and gives a robust assessment of each regressor's importance by considering different orderings of how they are added to the model. 
The iterative process for the regressors $\{X_0, X_1, X_2, X_3\}$, where $X_0$ is the intercept, would be
\begin{enumerate}
    \item Considering $\{X_1, X_2, X_3\}$,  $X_1$ is added to the model, and the share of explained variance allocated to $X_1$ is $\text{svar}(\{1\} \lvert \emptyset)$. $X_2$ is added and allocated a share of $\text{svar}(\{2\} \lvert \{1\})$, and lastly $X_3$ is added and allocated a share of $\text{svar}(\{3\} \lvert \{1, 2\})$.
    \item Considering $\{X_1, X_3, X_2\}$,  $X_1$ is added to the model, and the share of explained variance allocated to $X_1$ is $\text{svar}(\{1\} \lvert \emptyset)$. $X_3$ is added and allocated a share of $\text{svar}(\{3\} \lvert \{1\})$, and lastly $X_2$ is added and allocated a share of $\text{svar}(\{2\} \lvert \{1, 3\})$.
\end{enumerate}
The above iteration is repeated for all 6 possible permutations of orderings among regressors to obtain the final result.
This iterative process gives rise to the general formula for share of explained variance allocated to $X_1$ by the LMG method with $p$ regressors \citep{gromping_relaimpo},
\begin{equation}
    \label{eq:LMG}
    \text{LMG}(1) = \frac{1}{p!} \sum_{S \subseteq \{2, ..., p\}} n(S)! (p - n(S)-1)! \text{svar}(\{1\} \lvert S) \ ,
\end{equation} 
where $n(S)$ is the number of regressors in $S$.
Equation \eqref{eq:LMG} averages the increase in $R^2$, $\text{svar}(\{X_i\})$, when adding the covariate of interest, $X_i$, over all possible orderings of covariates. 
This mean increase over orderings is assigned as the proportion of $R^2$ explained by $X_i$.
The LMG method fulfills all but the exclusion criteria described previously \citep{gromping_relaimpo}, but \citet{gromping_relaimpo} argues that this "must be seen as a natural result of model uncertainty" and therefore that this criterion is not indispensable.
Therefore, we find it also suitable for our purposes to focus on the three other criteria.
The setback of the LMG method is the great computational expense that the permutations require when $p$ is large. The complexity is $2^{p-1}$ summations \citep{gromping_relaimpo}, and therefore, the LMG is not suitable for high dimensional models. 



\subsection{Relative weights method}
\label{sec:relativeweights}
A method that takes advantage of the straightforward decomposition of the variance when the fixed covariates are uncorrelated is the relative weights method \citep{johnson_relative_weights}, which will now be discussed.
\newline
\newline
The relative weights method proposes an alternative to the LMG, which is significantly less computationally expensive. 
Intuitively, the relative weights method projects the design matrix $\mathbf{X}$ of the fixed effects into an orthogonal column space, resulting in a matrix $\mathbf{Z}$ with orthogonal columns.
The matrix $\mathbf{Z}$ is then an approximation of $\mathbf{X}$ and will be used as the design matrix in the regression. Since the columns of the design matrix $\mathbf{Z}$ are orthogonal, each covariate is uncorrelated. 
This allows us to decompose the variance in the straightforward manner as in equation \eqref{eq:var_y_uncorrelated}.
% As is mentioned above, decomposing the variance with uncorrelated predictors is straightforward. To see this, one can with out loss of generalization, standardize the repsonse to have the zero-vector as mean and the identity matrix as variance. Then it is clear that the marginal variance can be found as
% \begin{equation}
%     \text{Var}(\mathbf{y}) = \text{Var}(\mathbf{X}^T\boldsymbol{\beta} + \boldsymbol{\varepsilon}) = \boldsymbol{\beta}^T\text{Var}(X)\boldsymbol{\beta} + \sigma^2 = \boldsymbol{\beta}^T\boldsymbol{\beta} + \sigma^2 \ ,
% \end{equation}
% so for each predictor the contribution to the full explained variance is the corresponding regression coefficient squared. This property lies the foundation for the relative weights method, which will now be considered.
\newline
\newline
In relative weights one uses the singular value decomposition \citep{relative_weights_nimon_oswald}, to project the real-valued design matrix $\mathbf{X}$ into an orthonormal matrix $\mathbf{U} \in \mathbb{R}^{n \times n}$ containing the eigenvectors of $\mathbf{X}\mathbf{X}^T$, an $n\times p$ diagonal matrix $\mathbf{D}$ containing the singular values of $\mathbf{X}$ and another orthonormal matrix $\mathbf{V} \in \mathbb{R}^{p \times p}$ containing the eigenvectors of $\mathbf{X}^T\mathbf{X}$ such that
%The above matrix relations differ from Matre, double check them!
\begin{equation}
    \label{eq:SVD}
    \mathbf{X} = \mathbf{UDV^T} \ .
\end{equation}
From the Eckhart-Young-Mirsky theorem \citep{mirsky-theorem} and following the derivations of \citet{johnson_minimization_trace}, one can state that the matrix $\mathbf{X}$, of rank $r$, can be approximated by a matrix $\mathbf{Z} = \mathbf{U}\mathbf{V}^T$ of rank $k\leq r$ such that the difference under the squared Frobenius norm
\begin{equation}
    \lVert \mathbf{X} - \mathbf{Z} \rVert_F^2 = tr \left( (\mathbf{X} - \mathbf{Z})^T(\mathbf{X} - \mathbf{Z}) \right) \ ,
\end{equation}
is minimized. The relative weights approximation now utilizes the matrix \citep{johnson_relative_weights} $\frac{1}{\sqrt{n-1}}\mathbf{{Z}}$, where the factor $\frac{1}{\sqrt{n-1}}$ is the standardization factor for $\mathbf{Z}$ \citep{matre}, and regresses on $\mathbf{Z}$ to find the MLE $\boldsymbol{\beta_Z}$ as
\begin{equation}
    \begin{aligned}
        \boldsymbol{\beta_Z} & = (\mathbf{Z}^T\mathbf{Z})^{-1}\mathbf{Z}\mathbf{y} \\
        & = \left((n-1) \mathbf{VU}^T\mathbf{UV}^T \right)^{-1} \sqrt{n-1}\mathbf{VU}^T\mathbf{y} \\
        & = \frac{1}{\sqrt{n-1}}\mathbf{V}\mathbf{U}^T\mathbf{y} \ .
    \end{aligned}
\end{equation} 
As $\mathbf{Z}$ is orthogonal, the relative importance for each column $\mathbf{z_i}$ with respect to the response $\mathbf{y}$ can be found as the square of $\beta_{Z, i}^2$, denoted as $\boldsymbol{\beta_Z}^{[2]}$. 
The notation $\boldsymbol{\xi}^{[2]}$ for some $\boldsymbol{\xi}$ represents the Schur product of $\boldsymbol{\xi}$ with itself, \textit{i.e.} element wise squaring of each element in $\boldsymbol{\xi}$.
Once these importances are obtained, \citet{johnson_relative_weights} argues that we should regress $\mathbf{X}$ on $\mathbf{Z}$ to obtain the weights that relate the importance of each column of $\mathbf{Z}$ to each column of $\mathbf{X}$. These weights can be calculated as the matrix
\begin{equation}
    \label{eq:lambda}
    \boldsymbol{\Lambda} = (\mathbf{Z}^T\mathbf{Z})^{-1}\mathbf{Z}^T\mathbf{X} = (\mathbf{V}\mathbf{U}^T\mathbf{U}\mathbf{V}^T)^{-1}\mathbf{V}\mathbf{U}^T\mathbf{U}\mathbf{D}\mathbf{V}^T = \mathbf{V}\mathbf{D}\mathbf{V}^T \ ,
\end{equation}
and since $\mathbf{Z}$ is orthogonal, the contribution from a column of $\mathbf{z_i}$ with respect to a column $\mathbf{x}_j$ is the squared entry $\boldsymbol{\Lambda}_{ij}^2$.
The contribution from a column $\mathbf{x}_j$ with respect to the response $\mathbf{y}$, \textit{i.e.} the relative importance, is then estimated as the matrix product \citep{johnson_relative_weights}
\begin{equation}
    \label{eq:RI_lambda}
    \text{RI}(\mathbf{X}) = \boldsymbol{\Lambda}^{[2]} \boldsymbol{\beta_Z}^{[2]} \ , 
\end{equation}
with $\text{RI}$ as a column vector where each entry $j$ contains the estimate of the relative importance corresponding to column $j$ of $\mathbf{X}$.
In \citet[section 2.5.3]{matre} it is shown that the relative weights method fulfills the criteria same three criteria as the LMG method, because $\mathbf{Z}$ and $\mathbf{X}$ are linear combinations of each other and due to the properties of $\boldsymbol{\Lambda}$.





% \newline
% \newline
% Due to the popularity of $R^2$, it is desirable to also be able to decompose the $R^2$ in the case of correlated covariates, such that it is consistent with the variance of the response.
% In (\ref{eq:var_y_full}) the response variance is split into three parts, the first two sums which comes from the regressors and the latter term which is the variance of the error. 
% It is the middle term that poses the problem of assigning importance to each covariate, since it contains the covariance between the covariates.
% The literature has established some conditions that relative importance measures should fulfill, so that they can be interpreted and compared in a sensible manner \citep{gromping_relaimpo}. As listed in \citet{gromping_relaimpo}, the methods should have
% \begin{enumerate}
%     \label{list:criteria}
%     \item \textbf{Proper decomposition}: The model variance should be decomposed into shares for each regressor that sum up to the total variance, and the method shall allocate the shares to each regressor.
%     \item \textbf{Non-negativity}: Each share of the variance should be non-negative.
%     \item \textbf{Exclusion}: If a regressor is excluded from the model, $\beta_j=0$, its share of the variance should be zero.
%     \item \textbf{Inclusion}: If a regressor is included in the model, $\beta_j \neq 0$, its share of the variance should be positive.
% \end{enumerate}
% Before moving on the such methods, some notation in accordance with \citet{gromping_relaimpo} will be introduced, namely
% \begin{equation}
%     \text{evar}(S) = \text{Var}(Y) - \text{Var}(Y \lvert X_j, j\in S) 
% \end{equation} 
% and
% \begin{equation}
%     \text{svar}(M \lvert S) = \text{evar}(M \cup S) - \text{evar}(S) \ ,
% \end{equation} 
% where $S$ is a subset of regressors, $\text{Var}(Y \lvert X_j, j\in S)$ denotes the variance of $Y$ conditioned on $X_j, j\in S$ being fixed, $\text{evar}(S)$ is the explained variance of the regressors in $S$ and $\text{svar}(M \lvert S)$ is the gain in variance explained by adding regressors from the subset $M$ to the model that already contains the regressors $S$. 

\section{Regression models}
The linear regression model is a popular tool in many sciences, but it has limitations when one wants to model more complex structures between the response and covariates. We now generalize the concept of linear regression to include more complex data structures.
\subsection{Generalized linear models (GLMs)}
\label{sec:GLM}
The first step in expanding the linear regression model, is to allow the responses to be non-Gaussian. 
%The linear regression and linear mixed model assumes that the response can be modelled as a linear combination of the fixed and random effects which, as we have seen, leads to the conditional distribution of the response being a normal distribution.
Instead of considering only the normal distribution as the distribution of the response, one can consider general responses belonging to the exponential family.
Assume that each we have $N$ observations of the response $y_{i}$, where $i=1, ..., N$, that are conditionally independent given the fixed effects.
Then, $y_{i}$ belongs to the univariate exponential family if
\begin{equation}
    \label{eq:exp_family}
    f(y_{i} \lvert \theta_{i}, \phi) = \exp\left(\frac{(y_{i}\theta_{i} - b(\theta_{i}))}{a(\phi)} + c(y, \phi) \right) \ ,
\end{equation}
for some functions $a(\cdot), b(\cdot)$ and $c(\cdot)$, where $\theta_{i}$ is the parameter of the distribution, $\phi$ is a dispersion parameter and $\theta_{i}$ is a canonical parameter if $\phi$ is known \citep{GLMM_book_old}. 
It is required that the function $b(\cdot)$ is twice differentiable, that the density function $f(y_{i} \lvert \theta_{i}, \phi)$ is normalizable and that the support of $f(y_{i} \lvert \theta_{i}, \phi)$ is not dependent on $\theta$.
Two key properties, expectation and variance, of the exponential family are given by
\begin{equation}
    \begin{aligned}
        \mathbb{E}(Y \lvert \theta) &= b'(\theta) \\
        \text{Var}(Y \lvert \theta) &= a(\phi) b''(\theta) \ ,
    \end{aligned}
\end{equation}
where $b''(\theta)$ may also be refered to as the variance function \citep{GLMM_book} we have left out indexing, and a proof can be found in \Cref{ap:proofs}.
In the canonical form, the parameter $\theta_{i}$ coincides with the linear predictor $\eta_{i}$ defined as 
\begin{equation}
    \theta_{i} = \eta_{i} = \mathbf{x}_{i}^T\boldsymbol{\beta}  \ . %= g(\mathbb{E}(y_{i, j} \lvert \theta_{i, j})) \ ,
\end{equation}
To connect the linear predictor $\eta_{i}$ to the response, we define a monotonic, differentiable link function $g(\cdot)$ such that
\begin{equation}
    \eta_{i} = g(\mu_{i}) = g(\mathbb{E}(Y_{i}))\ .
\end{equation} 
Some examples of link functions are the identity function for the linear regression, the logit and probit functions for the Bernoulli distribution and the log function for the Poisson distribution.

% Instead of modelling the response as having a linear relationsship with the covariates, one links the expected value of the response, $\mathbb{E}[Y]$ to the linear predictor $\mathbf{X}\boldsymbol{\beta}$ through a link function $g(\cdot)$. 




\subsection{Linear mixed models (LMMs) \protect\footnote{This subsection is the same as in the project thesis \citep{Arnstad}.}}
\label{sec:LMM}
Data often comes in clustered form, for example due to repeated measurements of the covariate over time. 
Clustered data violate with the assumption of independent responses in linear regression and must be properly accounted for. One solution to this is to introduce random effects that are cluster specific, but independent of the fixed effects and the other clusters. Let the population contain $m$ underlying clusters, with $n_j\ , j=1, ..., m$ observations in each cluster, so that $\mathbf{y} \in \mathbb{R}^{(N \times 1)}$ where $N = \sum_{j=1}^m n_j $. Assume that we investigate $q$ random effects, including a random intercept and $q-1$ random slopes, such that the random effects vector can be written as 
\begin{equation}
    \label{eq:alpha}
    \boldsymbol{\alpha} = (\boldsymbol{\alpha}_1, ..., \boldsymbol{\alpha}_{m})^T \ ,
\end{equation}
where each $\boldsymbol{\alpha}_j \in \mathbb{R}^{q \times 1}$ is assumed independent and represents the random effects for cluster $j$ and has length $q$. For a cluster $j$ the vector $\boldsymbol{\alpha}_j \sim \mathcal{N}_q(\mathbf{0}, \mathbf{\Sigma})=\mathcal{N}_q(\mathbf{0}, \mathbf{Q}^{-1})$ where $\mathbf{\Sigma}$ is the $q \times q$ unknown covariance for the random effects assumed to be positive definite and $\mathbf{Q} = \mathbf{\Sigma}^{-1}$ the corresponding precision matrix. 
If the random effects for each cluster are independent of each other, the covariance matrix $\mathbf{\Sigma} = \text{diag}(\sigma_0^2, ..., \sigma_q^2)$.
The linear mixed model now takes the form
\begin{equation} \label{eq:LMM}
    \mathbf{y} = \mathbf{X}\boldsymbol{\beta} + \mathbf{U}\boldsymbol{\alpha} + \boldsymbol{\varepsilon} \ ,
\end{equation}
where $\mathbf{X} \in \mathbb{R}^{N \times p}$ is the design matrix for the fixed effects, $\boldsymbol{\beta} \in \mathbb{R}^{p \times 1}$ are the regression coefficients for the fixed effects, $\mathbf{U} = \text{diag}(\mathbf{U_j}) \ , \in \mathbb{R}^{N \times q}$ is the design matrix for the random effects and $\mathbf{U}_j \in \mathbb{R}^{n_j \times q}$ is the design matrix for cluster $j$. 
Since $\boldsymbol{\alpha}$ is a random variable, the parameter to estimate is the variance of each random effect $\mathbf{\Sigma}_{kk}=\sigma_k^2$ and their covariance $\mathbf{\Sigma}_{k, l} = \sigma_{k, l}$, where $k, l =1, ..., q$.
In practice it is often easier to estimate the precision rather than the variance, so calculations often involve the precision matrix $\mathbf{Q}$ rather than the covariance matrix $\mathbf{\Sigma}$.
In this model the independence between clusters are conserved for the response as a whole, but it expresses the correlation that observations of the same cluster have through the random effects.
As for the simple linear regression it is assumed that $\mathbf{X}\boldsymbol{\beta}$ is fixed, and that $\mathbf{U}$ is given, so they do not contribute to the model's variance. Therefore, the conditional expectation $\mathbb{E}(\mathbf{y} \lvert \mathbf{X}, \mathbf{U}) = \mathbf{X}\boldsymbol{\beta}$ is easily obtained, and the conditional variance can be calculated as
\begin{equation} \label{eq:cond_var_LMM}
    \text{Var}(\mathbf{y} \lvert \mathbf{X}, \mathbf{U}) = \text{Var}(\mathbf{X}\boldsymbol{\beta}  + \mathbf{U}\boldsymbol{\alpha} + \boldsymbol{\varepsilon}) = \mathbf{U}\text{Var}(\boldsymbol{\alpha})\mathbf{U}^T + \sigma^2\mathbf{I} = \mathbf{U}\mathbf{G}\mathbf{U}^T + \sigma^2\mathbf{I} \ ,
\end{equation}
where $\mathbf{I}\in \mathbb{R}^{N\times N}$ and $\mathbf{G} \in \mathbb{R}^{mq \times mq}$ is the block diagonal covariance matrix of the random effects, with $\mathbf{\Sigma}_j$ along the diagonal for $j=1, ..., m$. 
As we assume that the random effects are independent of the fixed effects, and that the random error term is iid for each observation, the conditional distribution of $\mathbf{y}$ follows that of a sum of independent normal distributions, \textit{i.e.}
\begin{equation}
    \mathbf{y} \lvert \mathbf{X}, \mathbf{U} \sim \mathcal{N}_n(\mathbf{X}\boldsymbol{\beta}, \mathbf{U}\mathbf{G}\mathbf{U}^T + \sigma^2\mathbf{I}) \ .
\end{equation}

\subsection{Generalized linear mixed models(GLMMs)}
\label{sec:GLMM}
Now that we have expanded the linear regression in two different ways, the final step to complete the regression framework is to combine the LMM and GLM to obtain the GLMM. This is done by adding random effects to the linear predictor, such that
\begin{equation}
    \theta_{i, j} = \eta_{i, j} = \mathbf{x}_{i, j}^T\boldsymbol{\beta} + \mathbf{u}_{i, j}^T \boldsymbol{\alpha}_j \ , %= g(\mathbb{E}(y_{i, j} \lvert \theta_{i, j})) \ ,
\end{equation}
where $j =1, ..., m$ denotes the cluster and $i=1, ..., n_j$ denotes the observations in cluster $j$.
As the random effects account for clustered variance, the assumption of conditionally independent observations is now loosened to the assumption of conditionally independent clusters \cite{GLMM_book}.

% The linear mixed model can be generalized to include more complex relationships between the response and the covariates. 
% As seen, both for the linear regression and linear mixed model, the response is assumed to have a linear relationship with covariates, leading to a conditional normal distribution of the response.
% %The linear regression and linear mixed model assumes that the response can be modelled as a linear combination of the fixed and random effects which, as we have seen, leads to the conditional distribution of the response being a normal distribution.
% Instead of considering only the normal distribution as the distribution of the response, one can consider general responses belonging to the exponential family.
% Assume that each response $y_{i, j}$, where $j =1, ..., m$ denotes the cluster and $i=1, ..., n_j$ denotes the observations in cluster $j$, are conditionally independent given the fixed and random effects.
% Then, $y_{i, j}$ belongs to the univariate exponential family if
% \begin{equation}
%     \label{eq:exp_family}
%     f(y_{i, j} \lvert \theta_{i, j}, \phi) = \exp\left(\frac{(y_{i, j}\theta_{i, j} - b(\theta_{i, j}))}{a(\phi)} + c(y, \phi) \right) \ ,
% \end{equation}
% for some functions $a(\cdot), b(\cdot)$ and $c(\cdot)$, where $\theta_{i, j}$ is the parameter of the distribution, $\phi$ is a dispersion parameter and $\theta_{i, j}$ is a canonical parameter if $\phi$ is known \citep{GLMM_book_old}. 
% It is required that the function $b(\cdot)$ is twice differentiable, that the density function $f(y_{i, j} \lvert \theta_{i, j}, \phi)$ is normalizable and that the support of $f(y_{i, j} \lvert \theta_{i, j}, \phi)$ is not dependent on $\theta$.
% Two key properties, expectation and variance, of the exponential family are given by
% \begin{equation}
%     \begin{aligned}
%         \mathbb{E}(Y \lvert \theta) &= b'(\theta) \\
%         \text{Var}(Y \lvert \theta) &= a(\phi) b''(\theta) \ ,
%     \end{aligned}
% \end{equation}
% where $b''(\theta)$ may also be refered to as the variance function \citep{GLMM_book} we have left out indexing, and a proof can be found in \Cref{ap:proofs}.
% In the canonical form, the parameter $\theta_{i, j}$ coincides with the linear predictor $\eta_{i, j}$ defined as
% \begin{equation}
%     \theta_{i, j} = \eta_{i, j} = \mathbf{x}_{i, j}^T\boldsymbol{\beta} + \mathbf{u}_{i, j}^T \boldsymbol{\alpha}_j %= g(\mathbb{E}(y_{i, j} \lvert \theta_{i, j})) \ ,
% \end{equation}
% where $\mathbf{x}_{i, j}$ and $\mathbf{u}_{i, j}$ are the $i$-th columns of the submatrices $\mathbf{X}_{j}$ and $\mathbf{U}_{j}$ of the larger design matrices $\mathbf{X}$ and $\mathbf{U}$ respectively, for cluster $j$. If the model contains only fixed effects, it is referred to as a generalized linear model (GLM) and if contains both fixed and random effects it is called a generalized linear mixed model (GLMM).
% To connect the linear predictor $\eta_{i, j}$ to the response, we define a monotonic, differentiable link function $g(\cdot)$ such that
% \begin{equation}
%     \eta_{i, j} = g(\mu_{i, j}) = g(\mathbb{E}(Y_{i, j}))\ .
% \end{equation} 
% Some examples of link functions are the identity function for the LMM, the logit and probit functions for the Bernoulli distribution and the log function for the Poisson distribution.









\section{Extending $R^2$ to GLMMs}
\label{sec:R2_GLMM}
As we generalized the linear regression to LMMs, GLMs and GLMMs, it is desirable to also generalize the concept of the $R^2$ to be applicable to these models. This is fundamental to be able to propose a method for decomposing the $R^2$ and thereby assigning relative importance to covariates.
However, the task of determining the $R^2$, and decomposing it, is not a trivial task in the linear regression case and becomes even more complex in the case of GLMMs.
Many extensions have been proposed, but due to a variety of theoretical problems and/or computational difficulties, no consensus has been reached on a framework for calculating the $R^2$ for GLMMs \citep{nakagawa2013general}.
To get an overview of the status quo for $R^2$, we will follow the paper by \citet{nakagawa2013general} and go through the different components added to the linear regression to compose the GLMMs.
\subsection{$R^2$ for GLMs}
Recalling the definition of the $R^2$ from \Cref{eq:R2}, we now generalize this to the GLMs. 
To illustrate the generalization, consider the deviance $\mathcal{D}(\mathbf{y}\lvert \theta)$ function which is defined as twice the difference between the log likelihood of the \textbf{saturated model} and the log-likelihood of the model of interest \citep{GLMM_book_old}.
The saturated model denotes the model of the maximum achievable log likelihood, and therefore fits the data perfectly. 
For a linear regression, with $\theta=(\boldsymbol{\beta}, \sigma^2)$, we would therefore obtain
\begin{equation}
    \begin{aligned}
    \mathcal{D}(\mathbf{y}\lvert \hat{\theta}) &= -2\left( \ln(\mathcal{L}(\boldsymbol{\beta}, \sigma^2 \lvert \mathbf{y})) - \ln(\mathcal{\hat{L}}(\boldsymbol{\hat{\beta}}, \hat{\sigma^2} \lvert \mathbf{y}))  \right) = -2\left( l(\boldsymbol{\beta}, \sigma^2 \lvert \mathbf{y}) - l(\boldsymbol{\hat{\beta}}, \hat{\sigma^2} \lvert \mathbf{y})  \right) \\
    & = -2\left(-\frac{n}{2}\ln(2\pi\sigma^2) - \frac{1}{2\sigma^2} (\mathbf{y}-\mathbf{X}\boldsymbol{\beta})^T(\mathbf{y}-\mathbf{X}\boldsymbol{\beta}) + \frac{n}{2}\ln(2\pi\sigma^2)  \right) \\
    & = \frac{1}{\sigma^2} (\mathbf{y}-\mathbf{X}\boldsymbol{\beta})^T(\mathbf{y}-\mathbf{X}\boldsymbol{\beta}) \\
    & = 1-R^2 \ ,
    \end{aligned}
\end{equation}
where $\mathcal{\hat{L}}$ denotes the saturated model.
Optimally, it is desirable to have as small deviance as possible while at the same time having a model that is not too complex.
The best practice of the deviance is not as model fit, but rather model comparison, where one compares models through the reduction in deviance \citep{GLMM_book_old}.
Since the model of interest is nested within the saturated model, the deviance coincides with the likelihood ratio test. 
By comparing the model of interest to the \textbf{null model}, the simplest fit possible, one obtains for the linear regression
\begin{equation}
    \begin{aligned}
    \mathcal{D}(\mathbf{y}\lvert \hat{\theta})- \mathcal{D}(\mathbf{y}\lvert \theta_0) &= -2\left( l(\boldsymbol{\beta}, \sigma^2 \lvert \mathbf{y}) - l(\boldsymbol{\hat{\beta}}, \hat{\sigma^2} \lvert \mathbf{y})  \right) + 2\left( l(\boldsymbol{\beta}_0, \sigma^2_0 \lvert \mathbf{y}) - l(\boldsymbol{\hat{\beta}}, \hat{\sigma^2})  \right) \\
    &= -2\left( l(\boldsymbol{\beta}, \sigma^2 \lvert \mathbf{y})  -  l(\boldsymbol{\beta}_0, \sigma^2_0 \lvert \mathbf{y})   \right) \\
    & = - \frac{2}{2\sigma^2}\left( - (\mathbf{y}-\mathbf{X}\boldsymbol{\beta})^T(\mathbf{y}-\mathbf{X}\boldsymbol{\beta})  + (\mathbf{y}-\mathbf{\overline{y}})^T(\mathbf{y}-\mathbf{\overline{y}}) \right) \\
    & = 1 -  \frac{(\mathbf{y}-\mathbf{X}\boldsymbol{\beta})^T(\mathbf{y}-\mathbf{X}\boldsymbol{\beta})}{\sigma^2} \\
    &= R^2 \ .
    \end{aligned}
\end{equation}
This is the basis for the definitions of the generalization of $R^2$ to GLMs \citep{nakagawa2013general}, which primarily rely on a ratio of the maximum likelihood of the model of interest and null model.
In \citet{nakagawa2013general}, two different $R^2$ measures are proposed as
\begin{equation}
    R^2_G = \left[1- \left(\frac{\mathcal{L}_0}{\mathcal{L}_M}\right) ^{2/n} \right] \frac{1}{1-(\mathcal{L}_0)^{2/n}}
\end{equation}
and 
\begin{equation}
    R^2_D = 1 - \frac{-2\ln(\mathcal{L}_M)}{-2 \ln(\mathcal{L}_0)}
\end{equation}
where $n$ denotes the total sample size, $\mathcal{L}_0$ is the likelihood of the null model and $\mathcal{L}_M$ is the likelihood of the model of interest.
A problem with likelihood based $R^2$ measures is that when generalizing to the larger class of GLMMs, it is often desirable to do parameter estimation using the restricted maximum likelihood (REML) instead of the maximum likelihood (ML) \citep{GLMM_book}.
The REML estimator transforms the data, meaning that models cannot be compared when fitted, and therefore the proposed measure of $R^2$ is not applicable to the REML framework \citep{nakagawa2013general}.
However, the extension of the $R^2$ measure to the larger class GLMMs will also cover an extension to the GLMs, and is discussed further below in \Cref{sec:R2GLMM}.

\subsection{$R^2$ for LMMs and random slope models}
In the LMMs, as opposed to the linear regression, one wishes to estimate two or more variance components instead of only the residual error variance.
This increases complexity and makes the task of assigning relative importance to the covariates even more challenging.
Initially, a definition was proposed for the $R^2$ in the LMMs that included fixed effects separately and then estimated the reduction in each variance component \citep[refering to Raudenbush \& Bryk 1986, 1992]{nakagawa2013general}. 
This violated a key condition, as adding a covariate could decrease $\sigma^2_{\varepsilon}$ while at the same time increasing $\sigma^2_{\alpha}$, which can lead to a negative $R^2$.
To handle this problem, Snijders \& Bosker (1994) \citep{nakagawa2013general} proposed a new definition of the $R^2$, dividing it into two components $R^2_1$ and $R^2_2$.
Considering the simple random intercept model in scalar form;
\begin{equation}
    \label{eq:random_intercept}
    y_{i, j} = \beta_0 + \mathbf{x}_{i, j}^T\boldsymbol{\beta} + \alpha_{j} + \varepsilon_{i, j} \ ,
\end{equation}
where $y_{i, j}$ denotes the $i$th observation in cluster $j$, $\beta_0$ is the fixed intercept, $\mathbf{x}_{i, j}$ is the column vector containing the covariates for the $i$th observation in cluster $j$, $\boldsymbol{\beta}$ is the $p \times 1$ vector of fixed effects, $\alpha_{j}$ is the random intercept for cluster $j$ and $\varepsilon_{i, j}$ is the residual error for the $i$th observation in cluster $j$, the two $R^2$ components can be expressed in two ways, with the first being
\begin{equation}
    \begin{aligned}
    R_1^2 &= 1-\frac{\text{Var} (y_{i, j} - \hat{y}_{i, j})}{\text{Var} (y_{i, j})} = 1-\frac{\sigma_{\varepsilon}^2 + \sigma^2_{\alpha}}{\sigma_{\varepsilon 0}^2 + \sigma^2_{\alpha 0}} \\
    \hat{y_{i, j}} &= \beta_0 + \mathbf{x}_{i, j}^T\beta \ ,
    \end{aligned}
\end{equation}
where $\sigma_{\varepsilon 0}^2$ and $\sigma^2_{\alpha 0}$ denote the residual and random effect variances of the null model respectively \citep{nakagawa2013general} and $\hat{y}_{i, j}$ denotes the fitted value of observation $i$ in the $j$th cluster.
Similarily, the second component is defined as
\begin{equation}
    \begin{aligned}
    R_2^2 &= 1-\frac{\text{Var} (y_{j} - \hat{\bar{y_{j}}})}{\text{Var} (\overline{y_{j}})} = 1-\frac{\sigma_{\varepsilon}^2 + \sigma^2_{\alpha}/k}{\sigma_{\varepsilon 0}^2 + \sigma^2_{\alpha 0}/k} \\
    k &= \frac{M}{\sum_{j=1}^M \frac{1}{m_j}} \ ,
    \end{aligned}
\end{equation}
where $\overline{y_j}$ is the mean for each observed value of the $j$th cluster, $\hat{\bar {y_j}}$ is the mean of the fitted values for the $j$th cluster, $k$ is the harmonic mean of the number of observations per cluster, $m_j$ is the number of observations for the $j$th cluster and $M$ is the total number of clusters \citep{nakagawa2013general}. 
Note that we have formulated the above definitions in a notation corresponding to our previous formulation of the LMM, and therefore uses clusters in general, whereas \citet{nakagawa2013general} refers to a cluster as being individuals with repeated measurements.
% Litt usikker på om dette blir 100% riktig?
The reason for dividing the $R^2$ into two components, is that intuitively the $R^2_1$ measures the within cluster variance explained  and the $R^2_2$ measures the between cluster variance explained \citep{nakagawa2013general}.
However, three problems arise when using this definition of the $R^2$ for LMMs. 
Firstly, the $R^2_1$ and $R^2_2$ can decrease in large models, secondly, $R_1^2$ and $R_2^2$ have not been generalized to more complex LMMs with more than one random effect and lastly, it is not clear how to generalize the $R_1^2$ and $R_2^2$ to GLMMs \citep{nakagawa2013general}.
To overcome these obstacles, \citet{nakagawa2013general} proposes a new formulation of the $R^2$ measure. 
Consider a general random intercept model as defined in \Cref{sec:LMM}, with $q$ random intercepts, as
\begin{equation}
    \mathbf{y} = \mathbf{X}\boldsymbol{\beta} + \mathbf{U}\boldsymbol{\alpha} + \boldsymbol{\varepsilon} \ ,
\end{equation}
with the parameters of interest being $\boldsymbol{\beta}$ and the variance components $\sigma^2_{\varepsilon}$ and $\sigma^2_{i}$ for the $i=1, ..., q$ clusters.
Then define the variance of the fixed effects as 
\begin{equation}
    \sigma^2_f = \text{Var}(\mathbf{X}\boldsymbol{\beta}) = \boldsymbol{\beta}^T\text{Var}(\mathbf{X})\boldsymbol{\beta} \ ,
\end{equation}
and further define the $R^2$ for the LMM as
\begin{equation}
    R^2_{\text{LMM(m)}} = \frac{\sigma^2_f}{\sigma^2_f + \sum_{i=1}^{q}\sigma^2_{i} + \sigma^2_{\varepsilon}} \ .
\end{equation}
This definition of the $R^2_{\text{LMM}}$ represents the marginal $R^2_{\text{LMM}}$, denoted by $(m)$, as it measures the proportion of the variance explained by the fixed effects alone, whereas the conditional $R^2_\text{LMM}$ can be defined as 
\begin{equation}
    R^2_{\text{LMM(c)}} = \frac{\sigma^2_f + \sum_{i=1}^{q}\sigma^2_{i}}{\sigma^2_f + \sum_{i=1}^{q}\sigma^2_{i} + \sigma^2_{\varepsilon}} \ .
\end{equation}
By inspection it is clear that this definition will never lead to negative values of the $R^2_{\text{LMM}}$. 
It may occur that the $R^2_{\text{LMM}}$ value may decrease when adding more covariates to the model, although \citet{nakagawa2013general} argues that this is unlikely.
This definition now covers the random intercept model, but has not taken into account the possibility of having a LMM with a random slope. 
To further extend the $R^2$ to the random slope model, \citet{Johnson2014} proposes a method for computing the mean random effect variance.
Consider the simple random intercept and slope model,
\begin{equation}
    y_{i, j} = \beta_0 + \mathbf{x}_{i, j}^T\boldsymbol{\beta} + \alpha_{0, j} + \alpha_{1, j}x_{i, j} + \varepsilon_{i, j} \ ,
\end{equation}
where the same notation is used as in \eqref{eq:random_intercept} with $\boldsymbol{\alpha_j}=(\alpha_{0, j}, \alpha_{1, j})$ being the random effect, $\alpha_{0, j}$ denoting the random intercept and $\alpha_{1, j}$ now denoting the random deviation from the global slope $\beta_1$, for cluster $j$.
The general assumption on the random effects are that 
\begin{equation}
    \begin{aligned}
    \begin{pmatrix}
        \alpha_0 \\
        \alpha_{1}
    \end{pmatrix} 
    \sim \mathcal{N}\left( 
    \begin{pmatrix}
        0 \\
        0
    \end{pmatrix} \ ,
    \boldsymbol{\Sigma} =
    \begin{pmatrix}
        \sigma^2_{\alpha_0} & \sigma_{\alpha_0, \alpha_1} \\
        \sigma_{\alpha_0, \alpha_1} & \sigma^2_{\alpha_1}
    \end{pmatrix} \right)  \ ,
    \end{aligned}
\end{equation}
where $\sigma^2_{\alpha_0}$ and $\sigma^2_{\alpha_1}$ are the variances of the random intercept and random slope respectively, and $\sigma_{\alpha_0, \alpha_1}$ is the covariance between the random intercept and random slope.
Thus, we have three variance components of interest ($\frac{q(q+1)}{2}$ for $q$ random effects) to estimate. 
When inspecting the variance of the random part in the model, we see that it has a dependence on the covariates, as illustrated by 
\begin{equation}
    \begin{aligned}
    \text{Var}(\alpha_{0, j} + \alpha_{1, j}x_{i, j}) &= \text{Var}(\alpha_{0, j}) + 2x_{i, j} \text{Cov}(\alpha_{0, j}, \alpha_{1, j}) + x^2_{i, j}\text{Var}(\alpha_{1, j}) \\
    & = \sigma^2_{\alpha_0} + 2x_{i, j}\sigma_{\alpha_0, \alpha_1} + x^2_{i, j}\sigma^2_{\alpha_1} =: \sigma^2_{r, i, j} \ ,
    \end{aligned}
\end{equation}
where we define $\sigma^2_{r, i, j}$ as the variance of the random effect $\boldsymbol{\alpha}$ for observation $i$ in the $j$th cluster. 
The method proposed by \citet{Johnson2014} is to first estimate all the variance components, and then view the specific random effect as a normal mixture distribution of the random intercept and random slope.
This mixture distribution is characterized as having a common mean of zero, and, if all values of the associated covariate $x_{i, j}$ are unique, having $N$ different variances with $N$ being the total number of observations.
A mixture distribution with constant mean, has a variance which equals the mean of the individual variances in the distribution \citep[citing Behboodian 1970]{Johnson2014}. 
The proposed variance of the random effect $\boldsymbol{\alpha}$, is therefore the mean of the variance components in $\boldsymbol{\alpha}$, \textit{i.e.}
\begin{equation}
    \overline{\sigma^2_{r}} = \frac{1}{N}\sum_{j=1}\sum_{i=1} \left(\sigma^2_{r, i, j} \right) \ .
\end{equation}
This formulation can be generalized in the case of $q$ random effects, where each random effect has an associated design matrix $\mathbf{U_j}$ and covariance matrix $\mathbf{Q}$ as in \Cref{sec:LMM}, so that for each random effect $r$ we have
\begin{equation}
    \overline{\sigma^2_{r}} = \text{Tr}( \mathbf{U_j}\mathbf{Q}\mathbf{U_j}^T), \ \ r=1, ..., q \ .
\end{equation}
To finally obtain the proposed $R^2$ for the general LMM, \citet{Johnson2014} uses this estimate in the definition given by \citet{nakagawa2013general}, to obtain
\begin{equation}
    R^2_{\text{LMM(m)}} = \frac{\sigma^2_f}{\sigma^2_f + \sum_{r=1}^{q}\overline{\sigma^2_{r}} + \sigma^2_{\varepsilon}} \ ,
\end{equation}
and 
\begin{equation}
    R^2_{\text{LMM(c)}} = \frac{\sigma^2_f + \sum_{r=1}^{q}\overline{\sigma^2_{r}}}{\sigma^2_f + \sum_{i=1}^{q}\overline{\sigma^2_{r}} + \sigma^2_{\varepsilon}} \ ,
\end{equation}
as the marginal and conditional $R^2_{\text{LMM}}$ respectively. 
For the random intercept model with $\sigma_{r, i, j}^2 = \sigma^2_r$, this definition corresponds to the definition by \citet{nakagawa2013general} as 
\begin{equation}
    \overline{\sigma^2_{r}} = \frac{1}{N}\sum_{j=1}\sum_{i=1} \left(\sigma^2_{r, i, j} \right) = \sigma^2_{r, i, j}  = \sigma^2_{r} \ .
\end{equation}
The $R^2_{\text{LMM}}$ proposed by Johnson now lets us compute the $R^2$ for general LMMs, however it is argued in \citet{Johnson2014} whether the improved $R^2$ estimate by taking the random slope into account is worth the added complexity and computational cost.

\subsection{$R^2$ for GLMMs}
\label{sec:R2GLMM}
The final step towards a complete generalization for the $R^2$ value of regression models is to extend it to the GLMMs. 
When considering non-normal responses, the link function introduces an aspect not yet discussed, which is to define the residual variance.
One can divide the residual variance $\sigma^2_{\varepsilon}$ into three components, namely distribution specific variance, multiplicative dispersion and additive dispersion \citep{nakagawa2013general}.
The distribution specific variance is inherited from the link function used, and is therefore known before analysis is done. 
However, the multiplicative and additive dispersion is modelled to account for the variance present that exceeds the distribution specific variance, \textit{i.e.} overdispersion \citep{NakagawaSchielzeth2010}.
Therefore, one must specify upon implementation on what scale the overdispersion is to be modelled. 
The multiplicative dispersion, denoted by $\omega$, is overdispersion on the response (data) scale and modelled as a distinct parameter of the assumed distribution of the response $\mathbf{y}$ \citep{NakagawaSchielzeth2010}.
Conversely, the additive dispersion, denoted by $e$, is overdispersion on the latent scale and introduced to the model as an additional random effect in the linear predictor \citep{NakagawaSchielzeth2010}.
Defining the residual variance now depends on the choice of dispersion modelling, and is either defined as
\begin{equation}
    \sigma^2_{\varepsilon} = \omega \sigma^2_{d} 
\end{equation}
or 
\begin{equation}
    \sigma^2_{\varepsilon} = \sigma^2_{d} + \sigma^2_{e} \ ,
\end{equation}
for multiplicative and additive dispersion respectively.
With the residual variance defined, the generalization to of the $R^2$ to GLMMs (thereby also the GLMs) follows the same logic as the LMMs, and $R^2_{\text{GLMM (m)}}$ is defined as
\begin{equation}
    R^2_{\text{GLMM(m, m)}} = \frac{\sigma^2_f}{\sigma^2_f + \sum_{r=1}^{q}\overline{\sigma^2_{r}} + \sigma^2_{\varepsilon}} = \frac{\sigma^2_f}{\sigma^2_f + \sum_{r=1}^{q}\overline{\sigma^2_{r}} + \omega \sigma^2_{d}}\ ,
\end{equation}
and
\begin{equation}
    R^2_{\text{GLMM(m, a)}} = \frac{\sigma^2_f}{\sigma^2_f + \sum_{r=1}^{q}\overline{\sigma^2_{r}} + \sigma^2_{\varepsilon}} = \frac{\sigma^2_f}{\sigma^2_f + \sum_{r=1}^{q}\overline{\sigma^2_{r}} + \sigma^2_{d} + \sigma^2_{e}}\ ,
\end{equation}
where the same notation as before is used and the subscripts $(m, m)$ and $(m, a)$ denote the multiplicative and additive dispersion respectively.
The conditional $R^2_{\text{GLMM}}$ can be defined in a similar manner,
\begin{equation}
    R^2_{\text{GLMM(c, m)}} = \frac{\sigma^2_f + \sum_{r=1}^{q}\overline{\sigma^2_{r}}}{\sigma^2_f + \sum_{r=1}^{q}\overline{\sigma^2_{r}} + \omega \sigma^2_{d}}\ ,
\end{equation}
and
\begin{equation}
    R^2_{\text{GLMM(c, a)}} = \frac{\sigma^2_f + \sum_{r=1}^{q}\overline{\sigma^2_{r}}}{\sigma^2_f + \sum_{r=1}^{q}\overline{\sigma^2_{r}} + \sigma^2_{d} + \sigma^2_{e}}\ ,
\end{equation}
completing the generalization. 


%USING LIKELIHOOD R2 VALUES MEANS THAT ONE CANNOT COMPARE MODELS WHEN USING REML, FIND ROOM TO INCLUDE THIS!


%THIS ARTICLE \citet{NakagawaSchielzeth2010} DESCRBIES REPEATABILITY FOR NON-GAUSSIAN DATA, COULD BE USEFUL LATER



\section{The Bayesian framework}
So far, we have introduced statistical concepts without considering the framework in which they are used.
We now expand the theory to consider the Bayesian framework, which is the framework used in this thesis.
% \subsection{The frequentist framework}
% The traditional framework of statistics is known as the \textit{frequentist} framework, in which the data is considered in two parts $(y, x)$ where $y$ is thought to be an observation of the random variable $Y$ and $x$ is considered stationary covariates \citep{Cox_2006}.
% Typically, one is solely concerned with the distribution of $Y$ conditional on $x$ and uses the covariates $x$ to describe the system in which $Y$ is observed.
% Therefore, a model, or family of models, for the random variable $Y$ must be chosen, giving rise to the probability density function
% \begin{equation}
%     \label{eq:likelihood}
%     f_Y(y \lvert x, \boldsymbol{\theta}) \ ,
% \end{equation}
% where $\boldsymbol{\theta}$ is the unknown parameter vector containing the parameters belonging to the parameter space of the model \citep{Cox_2006}. 
% The parameter $\boldsymbol{\theta}$ defines key properties of the distribution such as expectation, variance and higher moments and are often the focus of the statistical analysis.
% Often, one can partition the model parameters into two parts, $\boldsymbol{\theta}=\psi, \lambda$, where $\psi$ denotes the parameters of interest and $\lambda$ a nuisance parameter \citep{Cox_2006}.
% The parameters are regarded as unknown constants, and the inference one makes is typically statements that specify a hypothetical long run probability of the parameters taking on certain values.
% To illustrate we use an example from \citet{Cox_2006};
% Consider the mean $\bar{Y}$ of a random variable $Y$ assumed to be normally distributed with observations $(y_1, ..., y_n)$, mean $\mu$ and variance $\sigma^2$.
% It can be shown that
% \begin{equation}
%     \label{eq:mean_normal}
%     \mathbb{P}(\bar{Y} > \mu - z \frac{\sigma}{\sqrt{n}}) = 1-c \ ,
% \end{equation}
% where $\Phi$ denotes the cumulative distribution of the standard normal distribution such that $\Phi(z) = 1-c$.
% This can be rearranged to give the statement 
% \begin{equation}
%     \label{eq:mean_normal_rearranged}
%     \mathbb{P}(\mu < \bar{Y} + z \frac{\sigma}{\sqrt{n}}) = 1 - c \ ,
% \end{equation}
% which \textit{can be interpreted as specifying a hypothetical long run of statements about $\mu$ a proportion $1-c$ of which are correct} \citep{Cox_2006}.
% The belief that the long run behaviour of the data can be used as a basis to draw conclusions is a fundamental idea of the frequentist framework.
\subsection{General idea} 
The Bayesian framework stems from the notorious theorem developed by Thomas Bayes, \citep{bayes_1763}, which states that for events $A$ and $B$, with nonzero probability of occuring, we have
\begin{equation}
    \label{eq:bayes_theorem}
    \mathbb{P}(A\lvert B) = \frac{\mathbb{P}(B \cap A)}{\mathbb{P}(B)} = \frac{\mathbb{P}(B\lvert A)\mathbb{P}(A)}{\mathbb{P}(B)} \ .
\end{equation}
This can be generalized to also apply to distributions of continuous random variables, namely that 
\begin{equation}
    \label{eq:bayes_theorem_func}
    \pi(\boldsymbol{\theta} \lvert \mathbf{y}) = \frac{\pi(\mathbf{y} \lvert \boldsymbol{\theta})\pi(\boldsymbol{\theta})}{\pi(\mathbf{y})} \ ,
\end{equation}
where $\pi(\boldsymbol{\theta} \lvert \mathbf{y})$ is called the posterior distribution of $\boldsymbol{\theta}$, $\pi(\mathbf{y} \lvert \boldsymbol{\theta})$ is the likelihood, or sampling, distribution of $\mathbf{y}$, $\pi(\boldsymbol{\theta})$ is the prior distribution of the parameters and $\pi(\mathbf{y}) = \int \pi(\mathbf{y} \lvert \boldsymbol{\theta}) \pi(\boldsymbol{\theta})$ is the marginal distribution of the data \citep{gelman2015Bayesian}.
In practice, the marginal distribution is often omitted and one only consider the proportionality of \eqref{eq:bayes_theorem_func}, i.e.
\begin{equation}
    \label{eq:bayes_theorem_func_prop}
    \pi(\boldsymbol{\theta} \lvert \mathbf{y}) \propto \pi(\mathbf{y} \lvert \boldsymbol{\theta})\pi(\boldsymbol{\theta}) \ .
\end{equation}
In the context of statistical analysis, with $\boldsymbol{\theta}$ being the parameter vector of the family of models for the random variable $Y$ under investigation, $\pi(\boldsymbol{\theta} \lvert \mathbf{y})$ is interpreted as the distribution of the parameters given the data $\mathbf{y}$.
This is the key element that separates the Bayesian framework from the frequentist framework, as the parameter $\boldsymbol{\theta}$ is now treated as random variable instead of being point estimates. 

\subsection{Prior and posterior distributions}
Generally, a Bayesian model is built by first introducing some prior knowledge through the prior distribution $\pi(\boldsymbol{\theta})$ and supplementing this with the likelihood function $\pi(\mathbf{y} \lvert \boldsymbol{\theta})$.
The prior distribution must be chosen based on the prior knowledge available, and can either be informative, noninformative or weakly informative \citep{gelman2015Bayesian}. 
As a compromise of the information in the prior and the likelihood of the data, the posterior distribution is obtained. 
The resulting posterior will be different from analysis to analysis, but some general relations between the prior and posterior are discussed in \citet{gelman2015Bayesian}.
In particular, it is stated that \textit{the posterior variance is on average smaller than prior variance by an amount that depends on the variation in posterior means over the distribution of possible data} \citep{gelman2015Bayesian}.
This further means that if one wishes to reduce the variability in the posterior, the potential for this lies in reducing the variation of possible posterior means.
The posterior distribution will therefore, in general, be a compromise between the prior and the likelihood, which with increasing sampling size will be increasingly influenced by the likelihood \citep{gelman2015Bayesian}.

\subsection{Penalising complexity (PC) priors}
Prior distributions pose a great feature by allowing for inclusion of prior information, but also a great challenge in that they must be chosen with care.
As the theory of this is wast and out of the scope for this thesis, we will be mostly concerned with the penalising complexity priors proposed in \citet{simpson2017penalising}.
In this paper, four main principles are desirable to follow when choosing a prior distribution, namely 
\begin{enumerate}
    \item \textbf{Occams razor} - If there is no evidence for a complex mode, a base model should be preferred. 
    \item \textbf{Measure of complexity} - The measure of model complexity is deifned as $d(f \lvert \lvert g) = \sqrt{2 \text{KLD}(f \lvert \lvert g)}$ where $\text{KLD}(f \lvert \lvert g)$ denotes the Kullback-Leibler divergence \citep[for more information]{simpson2017penalising}.
    \item \textbf{Constant rate penalisation} - The penalisation, i.e. the decay of prior mass, grows as the complexity grows, but it is desirable that this growth is constant.
    \item \textbf{User defined scaling} - Assuming that the user has an idea of the magnitude of the parameter of interest, the user should be able to scale the prior accordingly.
\end{enumerate}
The PC priors therefore pose interpretable, applicable priors which are consistent with the above principles, and are therefore a practical choice for the Bayesian framework.
Particularly, for the case of a linear mixed model with a Gaussian random effect $\alpha \sim \mathcal{N}(0, \sigma^2 \mathbf{R}) = \mathcal{N}(0, \tau^{-1} \mathbf{Q}^{-1})$, the base model of the PC priors corresponds to the case where the precision $\tau=0$ and the prior for $\tau$ takes the form
\begin{equation}
    \label{eq:PC_prior}
    \pi(\tau) = \frac{\lambda}{2} \tau^{-3/2} \exp\left(-\lambda \tau^{-1/2}\right), \ \   \tau, \lambda>0 \ .
\end{equation}
To specify $\lambda$, the user is required to supply the values $(U, a)$ such that $\mathbb{P}(1/\sqrt{\tau} > U) = a$. 
This defines the scaling parameter of principle $4$ and leads to $\lambda=-\ln{a}/U$ \citep{simpson2017penalising}.
When fitting additive models, thereby modelling additive overdispersion, using PC priors is a natural choice \citep{gomezrubio2020inla}.

\subsection{Hierarchical Bayesian modelling}
When modelling in the Bayesian framework, the posterior distribution of the parameter $\boldsymbol{\theta}$ given the data is what one wants to infer.
For many applications, $\boldsymbol{\theta}$ is a high dimensional vector, with naturally connected entries \cite{gelman2015Bayesian}.
It may therefore be reasonable to assume that the parameters themselves are drawn from a population distribution, which can further be modelled by what is called hyperparameters.
The main idea is that the prior $\pi(\boldsymbol{\theta})$ itself contains a hierarchical structure and can be split into levels of conditional prior distributions, i.e. $\pi(\boldsymbol{\theta}) =  \pi(\boldsymbol{\theta} \lvert \boldsymbol{\phi})\pi(\boldsymbol{\phi})$ \citep{robert2007bayesian}.
Assuming that the data $\mathbf{y}$ depends only on the parameter $\boldsymbol{\theta}$, and that $\boldsymbol{\theta}$ depends on the hyperparameters $\boldsymbol{\phi}$, we can write the joint posterior distribution of $(\boldsymbol{\theta}, \boldsymbol{\phi})$ as
\begin{equation}
    \pi(\boldsymbol{\theta}, \boldsymbol{\phi} \lvert \mathbf{y}) \propto \pi(\mathbf{y} \lvert \boldsymbol{\theta}, \phi) \pi(\boldsymbol{\theta} \lvert \boldsymbol{\phi})\pi(\boldsymbol{\phi}) = \pi(\mathbf{y} \lvert \boldsymbol{\theta}) \pi(\boldsymbol{\theta} \lvert \boldsymbol{\phi})\pi(\boldsymbol{\phi}) \ ,
\end{equation}
where $\pi(\boldsymbol{\phi})$ is a prior placed on the hyperparameters. 
This hierarchical structure allows us to first estimate the population distribution using the hyperparameters, and then estimate the parameters of interest using the population distribution, instead of estimating each component of $\boldsymbol{\theta}$ separately \citep{gelman2015Bayesian}.
It may be practical to view the model in three parts and consider an example with a tractable posterior distribution. 
Let the observational model be $\pi(\mathbf{y} \lvert \boldsymbol{\theta})$ be defined as 
\begin{equation}
    y_i \lvert \theta_i \sim \text{Po}(\theta_i) \ , i = 1, ..., n \ ,
\end{equation}
for conditionally independent observations $y_i$ given the parameters $\theta_i$. 
Define then the latent model $\pi(\boldsymbol{\theta} \lvert \boldsymbol{\phi})$ as
\begin{equation}
    \theta_i \lvert \phi \sim \text{Gamma}(\alpha, \beta) \ ,
\end{equation}
for conditionally independent parameters $\theta_i$ given the hyperparameters $\alpha, \beta$.
Lastly, consider the hyperpriors $\pi(\boldsymbol{\phi})$ as
\begin{equation}
    \alpha \sim \text{Exp}(a) \ \text{and} \ \beta \sim \text{Gamma}(b, c) \ ,
\end{equation}
The full posterior density now reads 
\begin{equation}
    \pi(\boldsymbol{\theta}, \alpha, \beta \lvert \mathbf{y}) \propto \underbrace{\prod^{n}_{i=1} \theta_i^{y_i}e^{-\theta_i}}_{\text{Po}(\theta_i)} \underbrace{\prod^{n}_{i=1}\frac{\beta^{\alpha}}{\Gamma(\beta)}\theta_i^{\alpha-1}e^{-\beta \theta_i}}_{\text{Gamma}(\alpha, \beta)} \underbrace{\alpha^{a-1}e^{-\alpha}}_{\text{Exp}(a)} \underbrace{\beta^{b-1}e^{-c\beta}}_{\beta \sim \text{Gamma}(b, c)} \ ,
\end{equation}
which can be used to make inference about the parameters of interest.
This hierarchical structure is similar to that of the GLMM and is therefore a natural way of modelling a Bayesian GLMM.
To set up a Bayesian GLMM, consider again the model in \eqref{eq:exp_family} with dispersion parameter $\phi$ and linear predictor
\begin{equation}
    \boldsymbol{\eta} = \mathbf{X}\boldsymbol{\beta} + \mathbf{U}\boldsymbol{\alpha} \ ,
\end{equation}
where we assume that $\boldsymbol{\alpha} \sim \mathcal{N}(0, \mathbf{Q}^{-1})$ for some precision matrix $\mathbf{Q} = \mathbf{Q}(\boldsymbol{\rho})$ dependent on the hyperparameter $\boldsymbol{\rho}$.
Then, to define the model, a prior must be assigned to the likelihood specific parameter $\phi$, the fixed effects coefficients $\boldsymbol{\beta}$, and the variance components of the random effects $\boldsymbol{\rho}$.
For a general GLMM belonging to the exponential family defined in \eqref{eq:exp_family}, the posterior can be written out as 
\begin{equation}
    \begin{aligned}
    &\pi(\boldsymbol{\beta}, \boldsymbol{\alpha}, \phi, \boldsymbol{\rho} \lvert \mathbf{y}) \propto \left(\prod_{j=1}^m \pi(\mathbf{y}_j \lvert \boldsymbol{\beta}, \boldsymbol{\alpha}, \phi, \boldsymbol{\rho}) \right) \pi(\boldsymbol{\alpha} \lvert \boldsymbol{\rho}) \pi(\boldsymbol{\beta}) \pi(\phi)\pi(\boldsymbol{\rho}) \ , \\
    & \propto \exp\left( -\frac{1}{2} \boldsymbol{\alpha}^T \mathbf{Q}(\boldsymbol{\rho}) \boldsymbol{\alpha} + \sum_{j=1}^{m} \ln \pi(\mathbf{y}_j \lvert \boldsymbol{\beta}, \boldsymbol{\alpha}, \phi) \right) \lvert \mathbf{Q}(\boldsymbol{\rho}) \rvert^{1/2} \pi(\boldsymbol{\beta}) \pi(\phi)\pi(\boldsymbol{\rho}) \ ,
    \end{aligned}
\end{equation}
where the vector $\mathbf{y}_j$ denotes the $j$th cluster of observations \citep{fong2010bayesian}.

\subsection{$R^2$ in the Bayesian framework\protect\footnote{This subsection is slightly modified from the project thesis \citep{Arnstad}.}}
\label{sec:bayes_R2}
When working in the Bayesian framework, the definition of $R^2$ for the linear regression is not as straightforward as in the classical framework. As parameters are not treated as fixed, but as random variables, the $R^2$ value will also be a random variable. A possible remedy to this could be to use the posterior mode of the parameters $\boldsymbol{\beta}$ in \eqref{eq:R2}, however \citet{gelman2017rsquared} states two conflicts that this poses. Firstly, the use of point estimates to calculate statistics in the Bayesian framework rejects the fundamental uncertainty of the Bayesian framework. Secondly, when the parameters are estimated in a Bayesian framework, there is no guarantee that the $R^2 \in [0, 1]$, reducing its intuitive interpretability. 
In \citet{gelman2017rsquared} a definition of the $R^2$ for the Bayesian linear regression is poposed. Consider a draw $s$ of the parameters $\boldsymbol{\beta}$ from the posterior distribution. Then, the proposed definition is
\begin{equation}
    \label{eq:bayes_r2}
    R_s^2 = \frac{\boldsymbol{\beta}_s^T \boldsymbol{\Sigma_{\mathbf{X^TX}}}\boldsymbol{\beta}_s}{\boldsymbol{\beta}_s^T \boldsymbol{\Sigma_{\mathbf{X^TX}}}\boldsymbol{\beta}_s + \sigma^2_s} \ ,
\end{equation}
where $\boldsymbol{\Sigma_{\mathbf{X^TX}}}$ is the covariance matrix of the design matrix $\mathbf{X}$ and $\sigma^2_s$ is the variance of the error term which can be sampled from the posterior distribution.
Contrary to the classical definition this definition of $R^2$ contains only the estimated values from our model and not the observed values. The reasoning behind this is to carry this inherent uncertainty in the Bayesian framework by not using point estimates from the posterior mean, but rather averaging over a posterior distribution. %Should I cite Gelman here?
Drawing enough samples from \eqref{eq:bayes_r2} one would eventually obtain also a distribution for the $R^2$ value.





% Considering the model
% \begin{equation}
%     \pi{\boldsymbol{\theta} \lvert \mathbf{y}} \propto \pi(\boldsymbol{\theta})\pi(\mathbf{y} \lvert \boldsymbol{\theta}) \ ,
% \end{equation}
% where $\boldsymbol{\theta}$ is the parameter of interest and $\mathbf{y}$ is the data.




% Add citations here, think the Fahrmeir book covers it
% The above discussion is rooted in the so called frequentist framework, implying that the parameters are treated as fixed and the uncertainty is quantified by the sampling distribution of the data. The Bayesian framework, on the other hand, treats the parameters as random variables and the uncertainty is quantified by the posterior distribution of the parameters.
% \subsection{General idea}
% The Bayesian framework is based on a generalization of Bayes theorem \citep[see Proposition 3]{bayes_1763} to functions, which states that 
% \begin{equation}
%     \label{eq:bayes_theorem}
%     \pi(\boldsymbol{\theta} \lvert \mathbf{y}) = \frac{\pi(\mathbf{y} \lvert \boldsymbol{\theta})\pi(\boldsymbol{\theta})}{\pi(\mathbf{y})} \ ,
% \end{equation}
% where $\pi(\boldsymbol{\theta} \lvert \mathbf{y})$ is the posterior distribution of the parameters $\boldsymbol{\theta}$ given the data $\mathbf{y}$, $\pi(\mathbf{y} \lvert \boldsymbol{\theta})$ is the likelihood function, $\pi(\boldsymbol{\theta})$ is the prior distribution of the parameters and $\pi(\mathbf{y})$ is the marginal distribution of the data. 
% These distributions give rise to many new perspectives and interpretations. Often one only considers the posterior in terms of being proportional to the product of the likelihood and prior, namely
% \begin{equation}
%     \pi(\boldsymbol{\theta} \lvert \mathbf{y}) \propto \pi(\mathbf{y} \lvert \boldsymbol{\theta})\pi(\boldsymbol{\theta}) \ .
% \end{equation}
% The proportionality is useful because the marginal distribution of the data is often intractable, and thus the posterior is only known up to a constant.
% From well established sampling methods, such as Markov Chain Monte Carlo, this is enough to eventually be able to sample effectively from the posterior distribution. If a prior is proposed, and one can express the likelihood, the posterior can be computed and also updated as more data becomes available. 
% \newline
% \newline
% The fundamental perspective of distributions instead of point estimates are what that separates the Bayesian framework from a frequentist setting, and allows for different interpretations.
% In natural sciences, measurements are often performed by professionals over a time period, and it is therefore useful to have a model that can adjust as more data becomes available. This is what the likelihood function allows for as it models the parameters as a function of the current data.
% Further, the prior allows for inclusion of some prior information about the parameters, which is often the case in natural sciences. These can be specified by experts in the field or by previous studies. Lastly, the fundamental uncertainty of the Bayesian framework is very useful.
% Since everything is modelled as a distribution, a corresponding variance is calculated. This variance can be a useful quantity for making statistical inference about the parameters one wishes to estimate. It also allows for capturing the fundamental uncertainty of measuring physical quantities. 

% \subsection{Bayesian hierarchical modelling}






% \subsection{Bayesian GLMMs}
% When one wants to apply the idea of linear mixed models in the Bayesian framework some key aspects change, and we will follow the logic of \citet{gelman2015Bayesian} to explain this theory in our setting. Considering a model as in (\ref{eq:LMM}) we have four parameters, namely $\boldsymbol{\beta}$, $\boldsymbol{\alpha}$, $\sigma^2_{\alpha}$ and $\sigma^2_{\varepsilon}$, where $\boldsymbol{\beta}$ and $\boldsymbol{\alpha}$ are model parameters dependent on $\sigma^2_{\alpha}$ and $\sigma^2_{\varepsilon}$ which are called hyperparameters.
% In a Bayesian framework these parameters are treated as random variables instead of values with a true, but unknown value, meaning that we must specify a distribution for the parameters. The posterior distribution of the model parameters will depend on the hyperparameters and the latent structure we assume the model to have. To define the prior distributions $\pi(\sigma^2_{\alpha})$ and $\pi(\sigma^2_{\varepsilon})$ of the hyperparameters, one assumes they are independent and chooses a distribution based on the prior information available. In this thesis we will use the Penalised Complexity priors, or PC priors \citep{simpson2017penalising}.
% If one assumes independence of the random effects and the fixed effects these priors will allow us, through methods discussed later in section \ref{sec:INLA_framework}, to derive marginal posterior distributions for the model parameters and sample from these. From these distributions we can obtain statistics such as posterior means and modes, posterior variances and credible intervals.
 
% \subsection{Appropriate definition of $R^2$ for the Bayesian framework}
% We wish to estimate relative importance in a Bayesian framework and report the distribution of $R^2$. To do this we must first consider how $R^2$ can be correctly defined and generalized in the Bayesian framework. 
% \subsection{$R^2$ for Bayesian linear regression}
% \label{sec:bayes_R2}
% When working in the Bayesian framework, the definition of $R^2$ is not as straightforward as in the classical framework. The classical definition of $R^2$ for linear regression is written as
% \begin{equation}
%     R^2 = 1 - \frac{\sum_{i=1}^{n}(y_i - \hat{y}_i)^2}{\sum_{i=1}^{n}(y_i - \overline{y})^2} \ ,
% \end{equation}
% where $\hat{y}_i$ is the predicted value of $y_i$ and $\overline{y}$ is the mean of the observed values of $y$. However, if one was to compare models based on this metric in the Bayesian framework, the denominator would not be fixed. With a variable denominator one cannot accurately interpret a change in $R^2$ value when comparing models. 
% \citet{gelman2017rsquared} proposed a definition of the $R^2$ for the Bayesian linear regression that will be considered in the following. Consider a draw $s$ of the parameters $\boldsymbol{\beta}$ from the posterior distribution. Then, the proposed definition is
% \begin{equation}
%     \label{eq:bayes_r2}
%     R_s^2 = \frac{\boldsymbol{\beta}_s^T \boldsymbol{\Sigma_{\mathbf{X^TX}}}\boldsymbol{\beta}_s}{\boldsymbol{\beta}_s^T \boldsymbol{\Sigma_{\mathbf{X^TX}}}\boldsymbol{\beta}_s + \sigma^2_s} \ ,
% \end{equation}
% where $\boldsymbol{\Sigma_{\mathbf{X^TX}}}$ is the covariance matrix of the design matrix $\mathbf{X}$ and $\sigma^2_s$ is the variance of the error term which can be sampled from the posterior distribution.
% Contrary to the classical definition this definition of $R^2$ contains only the estimated values from our model and not the observed values. The reasoning behind this is to carry this inherent uncertainty in the Bayesian framework by not using point estimates from the posterior mean, but rather averaging over a posterior distribution. %Should I cite Gelman here?
% Drawing enough samples from \eqref{eq:bayes_r2} one would eventually obtain also a distribution for the $R^2$ value.

% \subsection{$R^2$ for Bayesian LMM's}
% \label{sec:bayes_R2_LMM}
% Since the Bayesian framework allows us to sample from the posterior distributions of both random and fixed effects, one can extend the conditional and marginal $R^2$ proposed by \citet{gelman2017rsquared} to the LMM case. 
% The respective generalization can be found directly as
% \begin{equation}
%     \label{eq:R2_bayes_LMM_cond}
%     R_{s, \text{marg}}^2 = \frac{\boldsymbol{\beta}_s^T \boldsymbol{\Sigma_{\mathbf{X^TX}}}\boldsymbol{\beta}_s}{\boldsymbol{\beta}_s^T \boldsymbol{\Sigma_{\mathbf{X^TX}}}\boldsymbol{\beta}_s + \sigma_{\alpha, s}^2 + \sigma_{\varepsilon, s}^2} \ ,
% \end{equation} 
% and
% \begin{equation}
%     \label{eq:R2_bayes_LMM_marg}
%     R_{s, \text{cond}}^2 = \frac{\boldsymbol{\beta}_s^T \boldsymbol{\Sigma_{\mathbf{X^TX}}}\boldsymbol{\beta}_s + \sigma_{\alpha, s}^2}{\boldsymbol{\beta}_s^T \boldsymbol{\Sigma_{\mathbf{X^TX}}}\boldsymbol{\beta}_s + \sigma_{\alpha, s}^2 + \sigma_{\varepsilon, s}^2} \ ,
% \end{equation}
% where the subscript $s$ denotes samples from the marginal posteriors of the parameters in question, \textit{i.e.} $\boldsymbol{\beta}, \sigma_{\alpha}^2$ and $\sigma_{\varepsilon}^2$.
% These definitions of the $R^2$ highlight exactly the fundamental advantage of the Bayesian framework. Since the $R^2$ is also treated as a random variable, it has a distribution which can be used for statistical inference. 
% Moreover, one can relate the $R^2$ more directly to the frequentist framework by using the posterior means or modes from the distributions of $\boldsymbol{\beta}, \sigma_{\alpha}^2$ and $\sigma_{\varepsilon}^2$ in \eqref{eq:R2_LMM_conditional} and \eqref{eq:R2_LMM_marginal}. This approach can be favorable for comparing methods.




\section{The INLA framework\protect\footnote{This subsection is slightly modified from the project thesis \citep{Arnstad}.}}
\label{sec:INLA_framework}
As we have seen, the analytical posterior is possible to obtain for some hierarchical structures. 
However, in the case of GLMMs, the posterior distribution is not in general analytically tractable \citep{fong2010bayesian}. This calls for the use of numerical methods, such as Markov Chain Monte Carlo (MCMC) methods, to be able to sample from the posterior distribution. 
Such methods are computationally expensive, and require careful analysis to justify convergence and mixing of the Markov chains to the posterior distribution. Therefore it is desirable, under certain conditions, to look at other methods that are more computationally efficient.
In this thesis we will consider the Integrated Nested Laplace Approximation (INLA) method \citep{gomezrubio2020inla}.
\newline
\newline
The INLA method is an alternative to the classical Marko Chain Monte Carlo methods, that has significant advantages at the cost of assuming a certain structure.
In order to apply INLA, consider the vector of observations $\mathbf{y} = (y_1, ..., y_n)$, which may also contain missing values. 
Given an appropriate link function $g(\mu_i)=\eta_i$, we can model the observations as independent given the linear predictor
\begin{equation}
    \eta_i = \alpha + \sum_{j=1}^{n_{\beta}} \beta_j z_{ji} + \sum_{k=1}^{n_{f}} f^{(k)}(u_{ki}) + \varepsilon_i \ , \hspace{10mm} i=1, ..., n \ ,
\end{equation}
where $\alpha$ is the intercept, $\beta_j$ are the regression coefficients for the covariates $z_{ji}$, $f^{(k)}$ are random effects for the vector of covariates $\mathbf{\{u_{k}\}}_{k=1}^{n_f}$ and $\varepsilon_i$ is the error term.
This gives rise to the key assumption that the INLA method needs in order to be applicable, namely that the latent field $\mathbf{x}$, denoted as
\begin{equation}
    \mathbf{x} = (\eta_1, ..., \eta_n, \alpha, \beta_1, ..., \beta_{n}) \ ,
\end{equation}
is a Gaussian Markov Random Field (GMRF). Further, it is assumed that observations are independent given this latent field and the latent field is distributed according to some hyperparameters $\boldsymbol{\theta}$.
The structure of the GMRF is given by a precision matrix $\mathbf{Q(\theta)}$, which is sparse and can be represented by a graph $\mathcal{G} = (\mathcal{V}, \mathcal{E})$ (see \Cref{sec:animalmodelGMRF} for more details). 
This along with the assumed conditional independence makes computations very fast and is why INLA is effective.
Now, the posterior distribution of the latent field $\boldsymbol{x}$ is given by
\begin{equation}
    \pi(\boldsymbol{x}, \boldsymbol{\theta} \lvert \mathbf{y}) = \frac{\pi(\mathbf{y} \lvert \boldsymbol{x}, \boldsymbol{\theta}) \pi(\boldsymbol{x} \lvert \boldsymbol{\theta}) \pi(\boldsymbol{\theta})}{\pi(\boldsymbol{y})} \propto \pi(\mathbf{y} \lvert \boldsymbol{x}, \boldsymbol{\theta}) \pi(\boldsymbol{x} \lvert \boldsymbol{\theta}) \pi(\boldsymbol{\theta}) \ ,
\end{equation}
where $\pi(\mathbf{y} \lvert \boldsymbol{x}, \boldsymbol{\theta})$ is the likelihood, $\pi(\boldsymbol{x} \lvert \boldsymbol{\theta})$ is the posterior of the latent field and $\pi(\boldsymbol{\theta})$ is the prior.
Since it is assumed that observations are independent given the latent field, we can further express
\begin{equation}
    \pi(\mathbf{y} \lvert \boldsymbol{x}, \boldsymbol{\theta}) = \prod_{i\in \mathcal{I}} \pi(y_i \lvert x_i, \boldsymbol{\theta}) \ ,
\end{equation}
where the index set $\mathcal{I} \subset \{1, 2, 3, \ldots, n\}$ only includes actual observed data. %SHOULD I INCLUDE MORE HERE? 
The INLA method now attempts to estimate the marginals of the latent effects and the hyperparameters. These marginals are given by
\begin{equation}
    \label{eq:INLA_marginals}
    \pi(x_l \lvert \mathbf{y}) = \int \pi(x_l \lvert \boldsymbol{\theta}, \mathbf{y}) \pi(\boldsymbol{\theta} \lvert \mathbf{y}) d\boldsymbol{\theta} \ ,
\end{equation}
and 
\begin{equation}
    \label{eq:INLA_marginals_hyperparameters}
    \pi(\theta_k \lvert \mathbf{y}) = \int \pi(\boldsymbol{\theta} \lvert \mathbf{y}) d\boldsymbol{\theta}_{-k} \ ,
\end{equation}
\citep{gomezrubio2020inla} respectively, $\boldsymbol{\theta}_{-k}$ is the vector of hyperparameters excluding element $\theta_k$ and the latter integral is possible to integrate numerically due to the low dimension of $\boldsymbol{\theta}$ \citep{rue2009inla}. The approximations of these integrals are omitted, see \citet{rue2009inla} for the full details. 
Lastly, the joint posterior distribution can be approximated from the so-called Skew Gaussian Copula class, as specified in \citet{rue2021joint}, and allows for sampling from the joint distribution. 
The INLA method is implemented in the R-package \texttt{R-INLA} \citep{gomezrubio2020inla} and is used in this thesis to fit the models and draw from the obtained posteriors. We note that for the random effects INLA outputs the precision for the parameters involved, which is defined as the inverse covariance matrix. For the posterior marginal distribution of variance for the random effects the package has a function for transforming the precision marginal to the variance marginal. The priors used for the models in this thesis follow the recommendations of penalizing priors by \citet{simpson2017penalising}.
EXPAND ON INLA? SINCE SOME OF THE FEEDBACK STATED THAT IT WAS A BIT UNCLEAR.
\\ 
\\ 
%WHAT MORE CHAPTERS TO INCLUDE IN THE THEORY SECTION?

\section{The Animal Model and quantitative genetics}
\label{sec:animalmodel}
% This is mostly from Kruuk, but I notice a bit of a difference when compared to Stenslands Bayesian animal model. Should I rather base my theory on Stensland? 
An important application of GLMMs, which we will later analyse, is in the context of evolutionary biology and quantitative genetics. 
To introduce the animal model and biological terminology, the section will rely heavily on the work of \citet{Kruuk2004} and \citet{ConnerHartl2004}. 
The animal model is a mathematical model, used as a tool for quantitative genetic analysis in evolutionary biology where the aim is to explain the phenotypic variation in a population.
A phenotype is defined as \textit{the outward appearance of an organism for a given characteristic} \citep{ConnerHartl2004}, such as eye color, height or behavior. 
In an organism, the observed phenotypic trait is a result of the complex interaction between environment and genotype. 
The genotype of a trait can be defined as \textit{the diploid pair of alleles present at a given locus}, and is the outcome of genetic inheritance \citep{ConnerHartl2004}. 
As evolutionary biology seeks to explain diversity \citep{Kruuk2004}, a decomposition of the phenotypic variance is of great interest. 
The simplest partition is to define the phenotypic variance as the sum of the genetic variance and environmental variance \citep{ConnerHartl2004}. 
However, for species that mate with other individuals in the population rather than self-fertilize, it is common to further decompose the genetic variance into three parts. 
The \textbf{total phenotypic variance} can therefore be partitioned as
\begin{equation}
    \sigma^2_P = \sigma^2_G + \sigma^2_E = \sigma^2_A + \sigma^2_D + \sigma^2_I + \sigma^2_E \ ,
\end{equation}
where $\sigma^2_P$ is the total phenotypic variance, $\sigma^2_G$ is the \textbf{genetic variance}, $\sigma^2_E$ is the \textbf{environmental variance}, $\sigma^2_A$ is the \textbf{additive genetic variance}, $\sigma^2_D$ is the \textbf{dominance genetic variance} and $\sigma^2_I$ is the \textbf{interaction genetic variance} \citep{ConnerHartl2004}.
The parameter of interest in the animal model is the additive genetic variance $\sigma^2_A$ \citep{Kruuk2004}, as the additive genetic effects are the only effects directly transferred to the offspring from its parents \citep{ConnerHartl2004}.
Thus, the animal model aims to estimate $\sigma^2_A$ to gain inference on how changes in phenotypic values across generations occur, which is defined as phenotypic evolution \citep{ConnerHartl2004}.
The animal model can be stated as a generalized linear mixed model, by letting 
\begin{equation}
    \boldsymbol{\eta} = g(\boldsymbol{\mu}) = \mathbf{X}\boldsymbol{\beta} + \mathbf{U}\boldsymbol{\alpha}\ ,
\end{equation}
where $\boldsymbol{\mu}$ is the mean of the observations $\mathbf{y}$ of the phenotypic trait(s), $\boldsymbol{\eta}$ is the linear predictor, $\mathbf{X}$ the design matrix of the fixed effects, $\boldsymbol{\beta}$ the population coefficients, $\mathbf{U}$ the design matrix of the random effects and $\boldsymbol{\alpha}$ the vector of random effects.
%, and $\boldsymbol{\varepsilon}$ the vector of residuals.
One of the random effects in the animal model, $\boldsymbol{\alpha}_A \sim \mathcal{N}(0, \mathbf{G})$,  accounts for the additive genetic effect.
As in \Cref{sec:LMM}, we let $\mathbf{G}$ denote the covariance matrix of the random effect $\boldsymbol{\alpha}_A$, which in the animal model can be derived from the expected covariance between relatives \citep{Kruuk2004}.
This derivation can be done by considering the coefficient of coancestry, $\Theta_{i,j}$, defined as \textit{the probability that an allele drawn at random from an individual $i$ will be identical by descent to an allele drawn at random from individual $j$} \citep{Kruuk2004}. 
We use the coefficient of coancestry to define the expected covariance between relatives, or additive relationship matrix, as $\mathbf{A}_{i, j}=2\Theta_{i, j}$ and consequently $\mathbf{G}=\sigma^2_A\mathbf{A}$ \citep{Kruuk2004}. 
The parameters of interest are now the additive genetic variance $\sigma^2_A$ and the \textbf{heritability}, defined as the proportion of the total phenotypic variance that is present due to the additive genetic variance, $\sigma^2_A/\sigma^2_P$ \citep{Wilson2008}.
 

\section{The Animal Model as a GMRF}
\label{sec:animalmodelGMRF}
INLA is a powerful tool for fitting latent gaussian models (LGMs) as it provides a computationally efficient alternative to the traditional MCMC methods \citep{rue2009inla}.
To be applicable it relies heavily on the latent field, which is Gaussian, to possess the Markov property. 
If a Gaussian random variable $\mathbf{X}=(X_1, ..., X_n)$ possesses the Markov property it means that for some $i\neq j$, $X_i$ is independent of $X_j$ conditioned $X_{-i, j}$, where $X_{-i, j}$ denotes all other elements of $\mathbf{X}$ except $X_i$ and $X_j$ \citep{rue2009inla}.
This property readily visualized in a conditional independence graph, and for the animal model the pedigree structure derived from the family relation can be used as the conditional independence graph \citep[as cited in \citet{Stensland_GMRF_bayes_animal_model}]{Wermuth1983Graphical}.
The pedigree of a population is a directed acyclic graph (DAG) where each node represents an individual and the directed edges represent the parent-offspring relationship. 
This gives rise to the conditional independence graph, which can be found by inserting edges between parents that share offspring and removing the directions in the pedigree \citep{Wermuth1983Graphical}.
An individual(node) in this graph will therefore only have edges, meaning it is conditionally dependent on, its parents, the parent(s) of its offspring, and its offspring.
FIGUR AV DETTE?
The pedigree can also be used to construct the relatedness matrix $\mathbf{A}$, previously defined as the expected covariance between relatives, and the gives rise to the sparse precision matrix $\mathbf{Q}:=\mathbf{A}^{-1}$ which is needed for calculations.
As we consider each node as an individual, the corresponding variable of that node is its breeding value $\boldsymbol{\alpha}$ \citep{Stensland_GMRF_bayes_animal_model}. 

\subsection{Single and Multitrait Animal Model}
When modelling the observed phenotypic trait values of individuals in a population, caution must be made when modelling the genetic component of the trait using INLA.
If only one trait is considered, 
The breeding values $\boldsymbol{\alpha}$ of an individual represents the genetic component of an observed phenotypic trait. 
FINSIH THIS PART



% Would it be correct to interpret this as meaning that all the covariance is explained by relatedness, and that the additive variance is a scalar? Or is there a more complex covariance structure that might be present?
